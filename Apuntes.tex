

%-----------------------------------------------------------------------------------------------------  
%	INCLUSIÓN DE PAQUETES BÁSICOS.
%-----------------------------------------------------------------------------------------------------

\documentclass{article}
\usepackage{enumerate}
\usepackage{wrapfig}
\usepackage{graphicx}
\usepackage{tikz,pgfplots}
\pgfplotsset{compat=1.15}
\usepackage{mathrsfs}
\usetikzlibrary{arrows}
\usepackage{wrapfig}
\usepackage{cutwin}
\usetikzlibrary{quotes,angles}
\usepackage{changepage}
\usetikzlibrary{decorations.pathmorphing}
\usepackage[bottom]{footmisc}

%-----------------------------------------------------------------------------------------------------
%	SELECCIÓN DEL LENGUAJE
%-----------------------------------------------------------------------------------------------------

% Paquetes para adaptar Látex al Español:
\usepackage[spanish,es-noquoting, es-tabla, es-lcroman]{babel} % Cambia
\usepackage[utf8]{inputenc}                                    % Permite los acentos.
\selectlanguage{spanish}                                       % Selecciono como lenguaje el Español.

%flechassssss
\def\flechaSobreyectiva{\mathrel{\mkern16mu  \vcenter{\hbox{$\scriptscriptstyle+$}}%
		\mkern-25mu{\longrightarrow}}}
\def\flechaInyectiva{\mathrel{\mkern0mu  \vcenter{\hbox{$\scriptscriptstyle+$}}%
		\mkern-9mu{\longrightarrow}}}
\def\flechaBiyectiva{\mathrel{\mkern16mu  \vcenter{\hbox{$\scriptscriptstyle+$}}\mkern-25mu{\mathrel{\mkern0mu\vcenter{\hbox{$\scriptscriptstyle+$}}\mkern-9mu{\longrightarrow}}}}}
\def\xFlechaInyectiva #1{\mathrel{\ooalign{\thinspace\thinspace$\mapstochar\mkern5mu$\hfil\cr$\xrightarrow{#1}$\cr}}}
\def\xFlechaSobreyectiva #1{\mathrel{\ooalign{\hfil$\mapstochar\mkern5mu$\thinspace\thinspace\cr$\xrightarrow{#1}$\cr}}}
%llave a la derecha
\newenvironment{rcases}
{\left.\begin{aligned}}
	{\end{aligned}\right\rbrace}

%-----------------------------------------------------------------------------------------------------
%	SELECCIÓN DE LA FUENTE
%-----------------------------------------------------------------------------------------------------

% Fuente utilizada.
\usepackage{courier}                    % Fuente Courier.
\usepackage{microtype}                  % Mejora la letra final de cara al lector.

%-----------------------------------------------------------------------------------------------------
%	ESTILO DE PÁGINA
%-----------------------------------------------------------------------------------------------------

% Paquetes para el diseño de página:
\usepackage{fancyhdr}               % Utilizado para hacer títulos propios.
\usepackage{lastpage}               % Referencia a la última página. Utilizado para el pie de página.
\usepackage{extramarks}             % Marcas extras. Utilizado en pie de página y cabecera.
\usepackage[parfill]{parskip}       % Crea una nueva línea entre párrafos.
\usepackage{geometry}               % Asigna la "geometría" de las páginas.

\providecommand{\norm}[1]{\lVert#1\rVert}
\providecommand{\abs}[1]{\lvert#1\rvert}
% Se elige el estilo fancy y márgenes de 3 centímetros.
\pagestyle{fancy}
\geometry{left=3cm,right=3cm,top=3cm,bottom=3cm,headheight=1cm,headsep=0.5cm} % Márgenes y cabecera.
% Se limpia la cabecera y el pie de página para poder rehacerlos luego.
\fancyhf{}

% Espacios en el documento:
\linespread{1.1}                        % Espacio entre líneas.
\setlength\parindent{0pt}               % Selecciona la indentación para cada inicio de párrafo.

% Cabecera del documento. Se ajusta la línea de la cabecera.
\renewcommand\headrule{
	\begin{minipage}{1\textwidth}
		\hrule width \hsize
	\end{minipage}
}

% Texto de la cabecera:
\lhead{\docauthor}                          % Parte izquierda.
\chead{}                                    % Centro.
\rhead{IEDO}              % Parte derecha.

% Pie de página del documento. Se ajusta la línea del pie de página.
\renewcommand\footrule{
	\begin{minipage}{1\textwidth}
		\hrule width \hsize
	\end{minipage}\par
}

\lfoot{}                                                 % Parte izquierda.
\cfoot{}                                                 % Centro.
\rfoot{Página\ \thepage\ de\ \protect\pageref{LastPage}} % Parte derecha.

%----------------------------------------------------------------------------------------
%	MATEMÁTICAS
%----------------------------------------------------------------------------------------

% Paquetes para matemáticas:
\usepackage{amsmath, amsthm, amssymb, amsfonts, amscd} % Teoremas, fuentes y símbolos.

\usepackage[all]{xy} %para diagramas conmutativos

% Nuevo estilo para definiciones
\newtheoremstyle{definition-style} % Nombre del estilo
{5pt}                % Espacio por encima
{5pt}                % Espacio por debajo
{}                   % Fuente del cuerpo
{}                   % Identación: vacío= sin identación, \parindent = identación del parráfo
{\bf}                % Fuente para la cabecera
{.}                  % Puntuación tras la cabecera
{.5em}               % Espacio tras la cabecera: { } = espacio usal entre palabras, \newline = nueva línea
{}                   % Especificación de la cabecera (si se deja vaía implica 'normal')

% Nuevo estilo para teoremas
\newtheoremstyle{theorem-style} % Nombre del estilo
{5pt}                % Espacio por encima
{5pt}                % Espacio por debajo
{\itshape}           % Fuente del cuerpo
{}                   % Identación: vacío= sin identación, \parindent = identación del parráfo
{\bf}                % Fuente para la cabecera
{.}                  % Puntuación tras la cabecera
{.5em}               % Espacio tras la cabecera: { } = espacio usal entre palabras, \newline = nueva línea
{}                   % Especificación de la cabecera (si se deja vaía implica 'normal')

% Nuevo estilo para ejemplos y ejercicios
\newtheoremstyle{example-style} % Nombre del estilo
{5pt}                % Espacio por encima
{5pt}                % Espacio por debajo
{}                   % Fuente del cuerpo
{}                   % Identación: vacío= sin identación, \parindent = identación del parráfo
{\scshape}                % Fuente para la cabecera
{:}                  % Puntuación tras la cabecera
{.5em}               % Espacio tras la cabecera: { } = espacio usal entre palabras, \newline = nueva línea
{}                   % Especificación de la cabecera (si se deja vaía implica 'normal')

% Teoremas:
\theoremstyle{theorem-style}  % Otras posibilidades: plain (por defecto), definition, remark
\newtheorem{theorem}{Teorema}[section]  % [section] indica que el contador se reinicia cada sección
\newtheorem{corollary}[theorem]{Corolario} % [theorem] indica que comparte el contador con theorem
\newtheorem{lemma}[theorem]{Lema}
\newtheorem{proposition}[theorem]{Proposición}
\newtheorem*{question}{Cuestión}

% Definiciones, notas, conjeturas
\theoremstyle{definition-style}
\newtheorem{definition}{Definición}[section]
\newtheorem{conjecture}{Conjetura}[section]
\newtheorem*{note}{Nota} % * indica que no tiene contador
\newtheorem*{observation}{Observación} % * indica que no tiene contador
\newtheorem*{properties}{Propiedades}
\newtheorem*{comment}{Comentario}

% Ejemplos, ejercicios
\theoremstyle{example-style}
\newtheorem{example}{Ejemplo}[section]
\newtheorem{exercise}{Ejercicio}[section]

%-----------------------------------------------------------------------------------------------------
%	PORTADA
%-----------------------------------------------------------------------------------------------------

% Elija uno de los siguientes formatos.
% No olvide incluir los archivos .sty asociados en el directorio del documento.
%\usepackage{title1}
\usepackage{title2}
%\usepackage{title3}

%-----------------------------------------------------------------------------------------------------
%	TÍTULO, AUTOR Y OTROS DATOS DEL DOCUMENTO
%-----------------------------------------------------------------------------------------------------

% Título del documento.
\newcommand{\doctitle}{}%git clone https://github.com/manudroid19/ApuntesEstructuras.git
% Subtítulo.
\newcommand{\docsubtitle}{}
% Fecha.
\newcommand{\docdate}{5 \ de \ Noviembre \ de \ 2018}
% Asignatura.
\newcommand{\subject}{Introducción a las Ecuaciones Diferenciales Ordinarias}
% Autor.
\newcommand{\docauthor}{Manuel de Prada, Jorge Vázquez}
\newcommand{\docaddress}{Universidade de Santiago de Compostela}
\newcommand{\docemail}{lalala@gmail.com}

%-----------------------------------------------------------------------------------------------------
%	RESUMEN
%-------------------------------					----------------------------------------------------------------------

% Resumen del documento. Va en la portada.
% Puedes también dejarlo vacío, en cuyo caso no aparece en la portada.
%\newcommand{\docabstract}{}
\newcommand{\docabstract}{Apuntes de la materia impartida en la USC}

\makeatletter
\renewenvironment{proof}[1][\proofname]{\par
	\pushQED{\qed}%
	\normalfont \topsep6\p@\@plus6\p@\relax
	\list{}{%
		\settowidth{\leftmargin}{\quad:\hskip\labelsep}%
		\setlength{\labelwidth}{0pt}%
		\setlength{\itemindent}{-\leftmargin}%
	}%
	\item[\hskip\labelsep\itshape#1\@addpunct{:}]\ignorespaces
}{%
	\popQED\endlist\@endpefalse
}
\makeatother

\begin{document}

%\maketitle

%-----------------------------------------------------------------------------------------------------
%	ÍNDICE
%-----------------------------------------------------------------------------------------------------

% Profundidad del Índice:
%\setcounter{tocdepth}{1}

\newpage
\tableofcontents
\newpage

%----------------------------------------------------------------------------------------
%	Sección 1: Deficiones y teoremas
%----------------------------------------------------------------------------------------

\section{Tema 1.}

\subsection{Motivaciones, generalidades y ejemplos de ecuaciones diferenciales ordinarias.}
\begin{example}
	Un tren viaja a 90 km/h. A las 10:00 está en el kilómetro 22. ¿Dónde está a las 12:30?
\end{example}
\begin{proof}[Solución]
	Tomando $x(t)$ como la posición en el instante $t$, sabemos que $ x'(t)=90 $, por lo que $ x(t)=90t+k $. Además, $ x(10)=22 $, por lo que con la ecuación anterior, $ k=-878 $. 
	
	Es decir, $ x(t)=90t-878 $ y en particular $ x(12,5)= 247 $.
\end{proof}

\begin{example}
	Un tren está parado en una ciudad A y avanza con una velocidad $ v(t)=2170t+20t^3 $ en m/min hasta que llega a la velocidad de crucero, que es de 4500 m/min. Se pide hallar:
	\begin{enumerate}[\qquad a)]
		\item Distancia recorrida en 1 minuto.
		\item Tiempo que tarda en alcanzar la velocidad de crucero.
		\item Metros recorridos cuando se alcanza la velocidad de crucero.
		\item Kilómetros recorridos en 30 minutos.
		\item Hora a la que pasará por la ciudad B, situada a 243 km de distancia, saliendo a las 7:00.
	\end{enumerate}
\end{example}
\begin{proof}[Solución] \ 
	\begin{enumerate}[\qquad a)]
	\item[b)] Sea $x(t)$ la posición en el instante $t$. Tomando $v(t)=x'(t)=2170t+20t^3=4500$, despejando resulta $ t=2 $. 
	
	\item Buscamos $ x(1) $, para lo que primeros obtendremos la expresión general $ x(t) $ que cumpla la condición $ x'(t)=2170t+20t^3 $.
	
	La primitiva de esta expresión es $ x(t)=\frac{2170t^2}{2}+\frac{20t^4}{4}+k $. Como $ x(0)=0 $, $ k=0 $ y $ x(1)=1090 $.
	
	\item[c)] Trivial a partir de los apartados a) y b). $ x(2)=4420 $m.
	
	\item[d)] En los dos primeros minutos, hasta alcanzar la velocidad de crucero ya hemos visto que se recorren 4420m. Para los 28 restantes, la velocidad es constante por lo que planteamos la ecuación $ x'(t)=4500 $. 
	
	Por tanto, $ x(t)=4500t+k $. Sabiendo que $ x(2)=4420 $, resulta $ k=-4580 $ y $ x(t)=4500t-4580 $ para $ t>2 $. En particular, $ x(30)=130420 $m $ =130,420 $km.
	
	\item[e)] Sabemos por el apartado anterior que se alcanza el punto D (130420m) a los 30 minutos. Los 112,58 metros restantes se recorren en $ \frac{112,58}{4,5\text{m/min}}\simeq 25 $ minutos, por lo que el trayecto en total requiere de 55 minutos y la hora de llegada es las 7:55. 
	
	Otra manera rápida es despejar $ t $ en la ecuación $ x(t)= 243000=4500t-4580 $.
	\end{enumerate}
\end{proof}
\begin{example}
	$ x(t) $ es la cantidad de medicamento presente en el organismo en un instante $ t $. La cantidad disminuye proporcionalmente a la cantidad de producto presente en el organismo. En $ t_0 $, tomamos 500mg. En 1 hora, la cantidad de medicamento en el organismo es de 380mg.
	\begin{enumerate}[\qquad a)]
		\item ¿Qué cantidad quedará en el cuerpo en 6h?
		\item ¿Cada cuanto tiempo hay que tomar el medicamento para que haya entre 100mg y 700mg en el organismo? 
	\end{enumerate}
\end{example}
\begin{proof}[Solución] \ 
	\begin{enumerate}[\qquad a)]
		\item Como la cantidad disminuye proporcionalmente a la cantidad de medicamento en el cuerpo, tenemos que si $ x(t) $ es la cantidad de medicamento en un instante $ t $, $ x'(t)=-kx(t) $. Además sabemos que $ x(0)=500 $ y $ x(1)=380 $. Para obtener la expresión correspondiente, razonamos:

		Si $ x'(t)=x(t) $, la única opción es que se trate de la exponencial. Nuestra expresión es parecida, tratemos de cumplir la igualdad $ x'(t)=-kx(t) $. Observamos que se cumple para $ x(t)=e^{-kt} $. Ahora nos fijamos en el primer punto,  $ x(0)=500 $. Con  $ x(t)=500e^{-kt} $ cumplimos esa condición. Por último, nos queda hallar una $ k $ que cumpla $ x(1)=500e^{-k}=380 $. 
		
		Nos queda $ k=-\log(\frac{38}{50}) $ y por lo tanto $ x(t)=500e^{\log(\frac{38}{50})t} =500(\frac{19}{25})^t$. En consecuencia, $ x(6)=96,35 $mg.

		\item Queremos obtener el $ t $ que cumpla $ 100<x(t)<200 $, para que la dosis no baje de los 100mg prescritos y al tomar de nuevo la dosis de 500mg no supere los 700mg.

		$ x(t)=500(\frac{19}{25})^t$ es una exponencial decreciente en todo su dominio, $ (0, +\infty) $. Despejando $ x(t)=500(\frac{19}{25})^t=100$ obtenemos $ t=\log_{\frac{19}{25}}(\frac{1}{5})=5,86 $ y para $ x(t)=200 $, $ t=3,33 $. Por lo tanto, para mantener el medicamento entre los 100mg y 700mg debe renovarse la dosis pasadas entre 3,33 y 5,86 horas.
	\end{enumerate}
\end{proof}
BEGIN JORGE 4/2
\subsection{Concepto de EDO y solución.}
\begin{definition}
	Sea $f: A \subset \mathbb{R} \times \mathbb{R}^n \longrightarrow \mathbb{R}^n$ con $f(t, x)$ una aplicación. Llamamos \emph{Ecuación Diferencial Ordinaria} (EDO) de primer orden dada en forma normal relativa a la función $f$ a: 
	\[ x' = f(t, x) \]
	Se dice que es de primer orden porque solamente aparece la primera derivada y en forma normal ya que $x'$ está despejada (si no sería en forma implícita, es decir, $h(t, x, x')=0$).
\end{definition}

\begin{definition}
	Dada $\varphi: t \in I \subseteq \mathbb{R} \longrightarrow \varphi (t) \in \mathbb{R}^n$ decimos que $\varphi$ es \emph{solución} de $x'=f(t, x)$ si verifica:
	\begin{enumerate}[\quad i)]
		\item $(t, \varphi (t)) \in A$, $\forall t \in I$.
		\item Existe $ \varphi' (t)$, $\forall t \in I$.
		\item $\varphi'(t)=f(t, \varphi(t))$,  $\forall t \in I$.
	\end{enumerate}
	Si $I$ contiene a uno de sus extremos entenderemos en el punto $2$ que exista su derivada lateral.
\end{definition}
\begin{observation}
	Tenemos que $\varphi: t \in \mathbb{R} \longrightarrow \mathbb{R}^n$ con $\varphi (t) = (\varphi_1 (t), ..., \varphi_n (t))$. Por tanto, que $\varphi' (t) = f(t, \varphi (t))$ significa lo siguiente:
	\[\varphi' (t) = (\varphi_1' (t), ..., \varphi_n' (t))\] 
	\[f(t, (x_1, ..., x_n)) = (f_1(t, (x_1, ..., x_n)), ..., f_n(t, (x_1, ..., x_n))) \]
	\[
	M=
	\left({\begin{array}{cc}
		\varphi_1' (t) \\
		\vdots \\
		\varphi_n' (t)
		\end{array} } \right)
	=
	\left({\begin{array}{cc}
		f_1(t, \varphi_1 (t), ..., \varphi_n (t)) \\
		\vdots \\
		f_n(t, \varphi_1 (t), ..., \varphi_n (t))
		\end{array} } \right)
	\]
	Es decir:
	\[\left . {\begin{array}{cc}
		\varphi_1' (t) = f_1(t, \varphi_1 (t), ..., \varphi_n (t)) \\
		\vdots \\
		\varphi_n' (t) = f_n(t, \varphi_1 (t), ..., \varphi_n (t))
		\end{array} } \right. \]
\end{observation}
\subsubsection{Continuidad y clase de la solución de una EDO.}
\begin{definition}
	Decimos que $ f:A\longrightarrow B $ es de clase $ r $, y escribimos $ f\in C^r(A) $ si sus derivadas parciales de orden $ r $ son continuas.
\end{definition}
\begin{proposition}
	Si $f$ es continua en $A$, entonces $\varphi$ es de clase $C^1$ en $I$.
\end{proposition}
\begin{proof} \ \\
	Por definición de solución $\varphi' (t) = f(t, \varphi (t))$, $t \in I$. Componiendo de la siguiente forma
	\[t \stackrel{g}{\longrightarrow} (t, \varphi (t)) \stackrel{f}{\longrightarrow} f(t, \varphi (t))\]
	tenemos que, tal y como está definida, $g$ es una función continua por estar compuesta de $id$, que es la identidad, y de $\varphi$, que es continua por ser derivable en todos sus puntos. Además, como $f$ es continua por definición y tenemos que $\varphi' (t) = f(t, \varphi (t))$, $\varphi'$ es continua por composición de continuas.
\end{proof}
\begin{theorem}
	Si $f \in C^r$, $r \in \mathbb{N}$ y $r \geq 1$, entonces $\varphi \in C^{r+1}(I)$.
\end{theorem}
\begin{proof}\ \\
	Realizaremos la prueba empleando inducción. En primer lugar lo probaremos para $r=1$, es decir, veamos que:
	\[f\in C^1(A) \Rightarrow \varphi \in C^2(I).\]
	Como $\varphi' (t) = f(t, \varphi (t))$, como hemos visto antes tenemos que:
		\[t \stackrel{g}{\longrightarrow} (t, \varphi (t)) \stackrel{f}{\longrightarrow} f(t, \varphi (t)),\]
	pero, tenemos que $f \in C^1(A)$ por lo que $f$ es continua y por la proposición anterior tenemos que $\varphi \in C^1$. Por lo tanto, la composición anterior cumple que es de clase $C^1$, por lo que concluímos que $\varphi \in C^2$. \\
	Supongamois ahora cierto para un $r$ arbitrario y probémoslo para $r+1$. Sea $f \in C^{r+1} (A)$ (por lo que $f \in C^r \stackrel{\text{hip.}}{\Rightarrow}$ $\varphi \in C^{r+1}$). Veamos que $\varphi \in C^{r+2}$.
	\[t \longrightarrow (t, \varphi (t)) \longrightarrow f(t, \varphi (t)) = \varphi' (t).\]
	En este caso, por el mismo razonamiento que en el caso $r=1$ tenemos que la composición es de clase $C^{r+1}$ por composición de funciones de clase $C^{r+1}$. Por lo tanto existe $\varphi^{r+2)}$ y es continua, por lo que $\varphi \in C^{r+2}$. 
\end{proof}
\subsubsection{Problema del valor inicial.}
Con el siguiente ejemplo introducimos la problemática de elegir correctamente el intervalo de definición de la solución, entre aquellos candidatos. Nos servirá para introducir la siguiente sección.
\begin{example}
	Sea $x' = x^2$, es decir, $f: (t, x) \in \mathbb{R} \times \mathbb{R} \longrightarrow f(t,x) = x^2 \in \mathbb{R}$. Comprobar que $x(t) = \frac{-1}{t+c}$, $c \in \mathbb{R}$, es solución. Estudiar en $ t=1 $.
\end{example}
\begin{proof}[Solución]\ \\
	En primer lugar tenemos que ver que $x'(t) = (x(t))^2 = f(t, x(t))$.
	\[x'(t) = \frac{1}{(t+c)^2} \Rightarrow x'(t) = (\frac{-1}{t+c})^2\]
	Además, de esta solución sabemos que:
	\[\left. {\begin{array}{cc}
		\lim\limits_{t \to (-c)^{-}} x(t) = \lim\limits_{t \to (-c)^{-}} \frac{-1}{t+c} = +\infty\\
		\lim\limits_{t \to (-c)^{+}} x(t) = \lim\limits_{t \to (-c)^{+}} \frac{-1}{t+c} = -\infty
		\end{array}} \right . \]
	por lo que la función tiene una asíntota vertical en $t = -c$. También sabemos que:
	\[\left. {\begin{array}{cc}
		\lim\limits_{t \to +\infty} x(t) = \lim\limits_{t \to +\infty} \frac{-1}{t+c} = 0^-\\
		\lim\limits_{t \to -\infty} x(t) = \lim\limits_{t \to -+\infty} \frac{-1}{t+c} = 0^+
		\end{array}} \right .\]
	por lo que tenemos una asíntota horizontal en $x = 0$. 
	
	
	\begin{figure}[h]
		\centering
		\begin{tikzpicture}[line cap=round,line join=round,>=triangle 45,x=1cm,y=1cm,scale=0.7]
		\begin{axis}[
		x=1cm,y=1cm,
		axis lines=middle,
		xmin=-2.2696161464380893,
		xmax=5.5453285056459745,
		ymin=-3.5608451100207597,
		ymax=3.87288272976652,
		xtick={-2,-1,...,5},
		ytick={-3,-2,...,3},]
		\clip(-2.2696161464380893,-3.5608451100207597) rectangle (5.5453285056459745,3.87288272976652);
		\draw[line width=1pt,smooth,samples=100,domain=-2.2696161464380893:5.5453285056459745] plot(\x,{0-1/((\x)-4/3)});
		
		\draw (1.371004410996194,0.7087831877032161) node[anchor=north west] {x=4/3};
		\end{axis}
		\end{tikzpicture}
		\caption{Solución del ejemplo para $ c=-4/3 $.} \label{E1}
	\end{figure}
	
	

	
	
	En este caso $A = \mathbb{R} \times \mathbb{R} = \mathbb{R}^2$. Además,
	\[\psi:t \in (- \infty, - c) \longrightarrow \psi(t) = \frac{-1}{t+c}\]
	\begin{enumerate}
		\item $(t, \psi(t)) \in A$
		\item $\exists \psi' (t) \ \forall t \in I$
		\item $\psi (t) = f(t, \psi (t))$
	\end{enumerate}
	Por lo tanto, $\psi$ es una solución de $x'=x^2$. Además,
	\[\varphi:t \in (-c, +\infty ) \longrightarrow \varphi(t) = \frac{-1}{t+c}\]
	también es solución por la misma razón que $\psi$, mientras que:
	\[t \in (- \infty, - c) \cup (-c, +\infty ) \longrightarrow f(t) = \frac{-1}{t+c}\]
	no es solución ya que no está definida en un intervalo. \\
	Además, si queremos calcular la solución de $x'=x^2$ que en $t =1$ valga $3$ tenemos que:
	\[3 = x(1) = \frac{-1}{1+c} \Rightarrow 3 + 3c = -1 \Rightarrow c = -\frac{4}{3}\]
	Por tanto, $x(t) = \frac{-1}{t-\frac{4}{3}}$ verifica que $x'(t) = (x(t))^2$, pasando por $ t=1 $. Además, por lo visto anteriormente, $\psi$ y $\varphi$ son soluciones en sus respectivos intervalos si hacemos que $c = - \frac{4}{3}$. Sin embargo, en este caso, solo $ \psi $ está definida en $ t=1 $.
\end{proof}
\subsection{Problema de Cauchy.}

\begin{definition}
	Dada $x' = f(t, x)$ con $f: A \subset \mathbb{R} \times \mathbb{R}^n \longrightarrow \mathbb{R}^n$ y un punto $(t_0, x_0) \in A$, el problema de Cauchy consiste en buscar una función:
	\[\varphi : I \subset \mathbb{R} \longrightarrow \mathbb{R}^n\]
	con $I$ un intervalo, que sea solución de $x' = f(t, x)$ tal que $\varphi (t_0) = x_0$.
\end{definition}

BEGIN MANU 6/2 exp
\begin{observation}
¿Una circunferencia puede ser solución de una EDO?
\end{observation}
\[ t^2+x^2=c \]
\begin{wrapfigure}{l}{3.5cm}   
	\begin{tikzpicture}[cap=round,>=latex,every node/.style={scale=0.5}]
	\draw[thick] (0cm,0cm) circle(1cm);
	\node[circle,draw=white, fill=white, inner sep=0pt,minimum size=5pt] (b) at (1cm,0) {};
	\node[circle,draw=white, fill=white, inner sep=0pt,minimum size=5pt] (b) at (-1cm,0) {};
	\draw[->] (-1.5cm,0cm) -- (1.5cm,0cm) node[right,fill=white]{$t$};
	\draw[->] (0cm,-1.5cm) -- (0cm,1.5cm) node[above,fill=white]{};
	\node[above left] at (-1cm,0) {$(-\sqrt{c},0)$};
	\node[above right] at (1cm,0) {$(\sqrt{c},0)$};
	\node[above right] at (0.7cm,0.7cm) {$x_1(t)$};
	\node[below right] at (0.7cm,-0.7cm) {$x_2(t)$};
	\end{tikzpicture}
\end{wrapfigure}
No, ya que no es una función. Sin embargo, si tomamos $ x_1=\sqrt{c-t^2} $ y $ x_1=\sqrt{c-t^2} $, son solución de la ecuación $ xx'+t=0 $.

Veámoslo para $ x_1 $.
\[ (c-t^2)^{\frac{1}{2}}\frac{1}{2} (c-t^2)^{-\frac{1}{2}}(-2t)+t=-t+t=0 \]
Su intervalo de definición será $ c-t^2\geq 0 \Rightarrow t \in [-\sqrt{c}, \sqrt{c}], c>0 $ y es derivable en $ (-\sqrt{c}, \sqrt{c}) $.

La solución escogida será una o otra en función de la condición inicial.
\[ t\in (-\sqrt{c}, \sqrt{c}) \longrightarrow \begin{cases}
x_1(t)=\sqrt{c-t^2}\\
x_2(t)=-\sqrt{c-t^2}
\end{cases}  \]

\pagebreak
\section{Tema 2.}

\begin{definition}
	Sea $ \varphi:t\in I \subset \mathbb{R}\longrightarrow \mathbb{R}^n $, con $ I $ intervalo, solución de la ecuación diferencial definida por $ f:(t,x) \in A \subset \mathbb{R}\times \mathbb{R}^n\longrightarrow f(t,x)\in \mathbb{R}^n$, tal que $ x'=f(t,x) $.
\end{definition}

Llamamos \emph{trayectoria} de $ \varphi $ al conjunto:
\[  \tau (\varphi)=\{ (t, \varphi(t)), t\in I\} \]
Si $ \varphi(t_0)=x_0 $, llamaremos \emph{semitrayectoria positiva} de $ \varphi $ al conjunto:
\[ \tau^+ (\varphi)=\{ (t, \varphi(t)) : t\in I, t>t_0\}  \]
Análogamente, la \emph{semitrayectoria negativa}  de $ \varphi $ será el conjunto:
\[ \tau^- (\varphi)=\{ (t, \varphi(t)) : t\in I, t\leq t_0\} \]

\begin{definition}
	Sea la ecuación diferencial definida por $ f:(t,x) \in A \subset \mathbb{R}\times \mathbb{R}^n\longrightarrow f(t,x)\in \mathbb{R}^n$, tal que $ x'=f(t,x) $.
	
	Sea $ \varepsilon\in \mathbb{R}, \varepsilon\geq 0 $, y sea $ \pi : I\subset \mathbb{R} \longrightarrow \mathbb{R}^n $ continua. Diremos que $ \pi $ es una \emph{solución $ \varepsilon $-aproximada} si verifica:
	
	\begin{enumerate}
		\item $ (t,\pi (t))\in A, \forall t \in I $.
		\item $ \pi $ continua y $C^1$ a trozos (es decir, es $C^1$ salvo a lo sumo una cantidad finita de puntos)
		\item $ || \pi'(t)-f(t, \pi(t))|| \leq \varepsilon$ para los puntos $ t\in I $ donde $ \pi $ es derivable. (Nótese que todas las normas son equivalentes en dimensión finita).
	\end{enumerate} 
\end{definition}
\begin{proposition}
	Sea la ecuación diferencial definida por $ f:(t,x) \in A \subset \mathbb{R}\times \mathbb{R}^n\longrightarrow f(t,x)\in \mathbb{R}^n$, tal que $ x'=f(t,x) $.
	
	Sea $ f $ continua y sea $ \pi : t \in I \subset \mathbb{R} \longrightarrow\mathbb{R}^n $, entonces:
	\begin{center}
		$ \pi  $ solución $ \Leftrightarrow $ $ \pi $ solución 0-aproximada.
	\end{center}
\end{proposition}
\begin{proof}\ \\
	"$ \Rightarrow $"
	
	$ \pi $	solución $\Rightarrow \begin{cases}
	i)\  (t, \pi (t)) \in A, \forall t \in I.\\
	ii)\  \text{Existe }  \pi' (t), \forall t \in I \Rightarrow \pi \in C^1(I). \\
	iii)\  \pi'(t)=f(t, \pi(t)),  \forall t \in I \Rightarrow \pi'(t)-f(t, \pi(t))=0 \Rightarrow \norm{\pi'(t)-f(t, \pi(t))}=0.
	\end{cases}$ 
	
	es decir, es solución 0-aproximada.
	
	"$ \Leftarrow $" \\Suponemos que $ \pi $ solución 0-aproximada y veamos que $ \pi $ es solución. En primer lugar tenemos que $(t, \pi (t)) \in A$, $\forall t \in I$, se verifica trivialmente. \\
	Sea $ \{t_i\}_{i=1}^k $ los puntos en los que $ \pi $ puede, en principio, no ser derivable, y veamos que existe la derivada en ellos. Por una parte, por ser  $ \pi $ solución 0-aproximada:
	 \[ ||\pi'(t)-f(t,\pi(t))||\leq 0\Rightarrow ||\pi'(t)-f(t,\pi(t))||=0 \Rightarrow \pi'(t)=f(t, \pi(t)), \forall t \in I, t\neq t_i.\]
	De lo que deducimos que cumple el punto iii) de la definición de solución. Además, fijado $ t_i\in \{t_1, \dots, t_k\} $, punto en el que $ \pi $ podría no ser derivable, como $ f $ y $ \pi $ son continuas, 
	\[ \lim\limits_{t\rightarrow t_i^-}\pi'(t)=\lim\limits_{t\rightarrow t_i^-} f(t, \pi(t))= f(t_i, \pi(t_i))= \lim\limits_{t\rightarrow t_i^+} f(t, \pi(t))=\lim\limits_{t\rightarrow t_i^+}\pi'(t)\]
	Y por lo tanto:
	\[ \begin{rcases}
	\lim\limits_{t\rightarrow t_i^-}\pi'(t)=\lim\limits_{t\rightarrow t_i^+}\pi'(t) \\
	\pi \text{ continua en } t_i
	\end{rcases}\Rightarrow \exists \pi'(t_i), \forall i \in \{1,\dots,k \} \]
	Con lo que se tiene que también cumple la condición ii).	
\end{proof}

\begin{theorem}
	Sea $ \pi: t  \in I \subset \mathbb{R} \longrightarrow\mathbb{R}^n $, con $ I  $ intervalo, una solución $ \varepsilon $-aproximada de la ecuación diferencial definida por $ f:(t,x) \in A \subset \mathbb{R}\times \mathbb{R}^n\longrightarrow f(t,x)\in \mathbb{R}^n$, tal que $ x'=f(t,x) $.
	
	Si $ f $ es continua, entonces:
	\[ ||\pi(t)-\pi(t_0)-\int_{t_0}^{t}f(s,\pi(s))ds||\leq \varepsilon |t-t_0|,\  \forall t, t_0 \in I\]
	 
\end{theorem}
\begin{proof}\ \\ 
	\textbf{Caso I.}
	
	Supongamos $ \pi \in C^1 $ salvo en los puntos $ t_1<\dots<t_k $.
	\begin{center}
		\begin{tikzpicture}[cap=round,>=latex,every node/.style={scale=1}]
			\foreach \x/\y in {0/t_1, 1/t_2, 4/t_k}
			\draw[thick] (\x,0.25) -- (\x,-0.25) node[below]{$\y$};
			\draw[thick] (-0.5,0)--(4.5,0);
			\draw (2.5,-0.25) -- (2.5,-0.25) node[below]{$\dots$};
		\end{tikzpicture}
	\end{center}
	
	
	Supongamos $ t,t_0 \in [t_i, t_{i+1}], i \in \{1, \dots, k-1\}$, es decir, supongamos que están en el mismo intervalo y que $ t_0<t $, sin perdida de generalidad:
	\begin{center}
		\begin{tikzpicture}[cap=round,>=latex,every node/.style={scale=1}]
			\foreach \x/\y in {0/t_i, 3/t_{i+1}, 6/t_{i+2}}
			\draw[thick] (\x/1.5,0.25) -- (\x/1.5,-0.25) node[below]{$\y$};
			\draw[very thick] (1/1.5,0.10) -- (1/1.5,-0.10) node[below]{$t_0$};	
			\draw[very thick] (2/1.5,0.10) -- (2/1.5,-0.10) node[below]{$t$};
			\draw[thick] (-0.5,0)--(4.5,0);
		\end{tikzpicture}
	\end{center}
	Sea $ J=[t_0,t], J \subset [t_i, t_{i+1}] $. Sea la función:
	\begin{align*}
	g:J&\longrightarrow \mathbb{R}^n\\
	u&\longrightarrow g(u)=\pi(u)-\pi(t_0)- \int_{t_0}^{u}f(s, \pi(s)) ds
	\end{align*}
	Sabemos que:
	\begin{enumerate}
	 \item $\pi$  continua por ser solución  $\varepsilon$-aproximada.\\
	 \item $s \longrightarrow (s, \pi(s))\longrightarrow f(s, \pi(s))$ es composición de continuas.\\
	 \item $F:J\longrightarrow \mathbb{R}$ que lleva $u \longrightarrow  \int_{t_0}^{u}f(s, \pi(s)) ds$ continua.
	\end{enumerate}
	Por lo tanto, podemos afirmar que $g$ es continua en $J$. Además $\pi$ es derivable en $\mathring{J}$ y $F$ es derivable en $J$, por lo que $g$ es derivable en $\mathring{J}$. Por el Teorema de los Incrementos Finitos\footnote{El Teorema de los Incrementos Finitos dice que dada cualquier función $f$ continua en el conjunto $A$ y diferenciable en $\mathring{A}$ entonces existe al menos algún punto $\xi$ en $\mathring{A}$ tal que $\norm{f(t)-f(t_0)}\leq \sup\norm{f'(\xi)}|t-t_0|$.}:
	\[\norm{g(t)-g(t_0)}\leq \sup \norm{g'(\xi)}|t-t_0|\]
	y, por la definición de $g$,
	\begin{align*}
	\norm{\pi(t)-\pi(t_0)-\int_{t_0}^{t} f(s, \pi(s)) ds -(\pi(t_0)-\pi(t_0)-\int_{t_0}^{t_0} f(s,\pi(s)) ds)} =\\ =\norm{\pi(t)-\pi(t_0)-\int_{t_0}^{t} f(s, \pi(s)) ds} \leq \sup \norm{g'(\xi)} |t-t_0|
	\end{align*}  
	Ahora bien, por el Teorema Fundamental del Cálculo y la definición de solución $\varepsilon$-aproximada: 
	\begin{align*}
	g'(u)=\pi'(u)-f(u,\pi(u))\Rightarrow ||g'(u)||=||\pi'(u)-f(u,\pi(u))||\leq \varepsilon 
	\end{align*}
	Y finalmente:
	\[ \begin{rcases}
	||\pi(t)-\pi(t_0)-\int_{t_0}^{t} f(s, \pi(s)) ds|| \leq \sup ||g'(\xi)|| |t-t_0|\\
	||g'(u)||=||\pi'(u)-f(u,\pi(u))||\leq \varepsilon 
	\end{rcases} \Rightarrow	||\pi(t)-\pi(t_0)-\int_{t_0}^{t} f(s, \pi(s)) ds|| \leq \varepsilon |t-t_0|\]
	
	BEGIN JORGE 11/2
	
	\textbf{Caso II.}
	En este caso el caso a analizar va a ser el de dos puntos en dos intervalos diferentes pero consecutivos. Es decir:
	\begin{center}
		\begin{tikzpicture}[cap=round,>=latex,every node/.style={scale=1}]
			\foreach \x/\y in {0/t_i, 3/t_{i+1}, 6/t_{i+2}}
			\draw[thick] (\x/1.5,0.25) -- (\x/1.5,-0.25) node[below]{$\y$};
			\draw[very thick] (1/1.5,0.10) -- (1/1.5,-0.10) node[below]{$t_0$};	
			\draw[very thick] (4/1.5,0.10) -- (4/1.5,-0.10) node[below]{$t$};
			\draw[thick] (-0.5,0)--(4.5,0);
		\end{tikzpicture}
	\end{center}
	Así, en este caso, tenemos que $t_o \in [t_i, t_{i+1}]$ y que $t \in [t_{i+1}, t_{i+2}]$. Por lo tanto, definiremos $J_j = [t_o, t_{i+1}]$ y $J_r = [t_{i+1}, t]$. Repitiendo el proceso en cada intervalo respecto al extremo común, se tiene que:
	\begin{enumerate}
		\item $\norm{\pi (t) - \pi (t_{i+1}) - \int_{t_{i+1}}^{t} f(s, \pi(s))ds} \leq \varepsilon |t - t_{i+1} |$ en $J_r$.
		\item $\norm{\pi (t_{i+1}) - \pi (t_0) - \int_{t_0}^{t_{i+1}} f(s, \pi(s))ds} \leq \varepsilon |t_{i+1} - t_0|$ en $J_j$.
	\end{enumerate}
	Consideramos ahora la suma siguiente:
	\[\pi (t) - \pi (t_{i+1}) - \int_{t_{i+1}}^{t} f(s, \pi(s))ds + \pi (t_{i+1}) - \pi (t_0) - \int_{t_0}^{t_{i+1}} f(s, \pi(s))ds = \pi (t) - \pi (t_0) - \int_{t}^{t_0} f(s, \pi(s))ds \]
	Por otro lado sabemos que:
	\[\norm{\pi (t) - \pi (t_{i+1}) - \int_{t_{i+1}}^{t} f(s, \pi(s))ds + \pi (t_{i+1}) - \pi (t_0) - \int_{t_0}^{t_{i+1}} f(s, \pi(s))ds} \stackrel{Des. Triangular}{\leq} \]
	\[\norm{\pi (t) - \pi (t_{i+1}) - \int_{t_{i+1}}^{t} f(s, \pi(s))ds} + \norm{\pi (t_{i+1}) - \pi (t_0) - \int_{t_0}^{t_{i+1}} f(s, \pi(s))ds} \stackrel{1. \text{ y } 2.}{\leq}  \varepsilon |t - t_{i+1} | + \varepsilon |t_{i+1} - t_0| = \varepsilon |t-t_0|\]
	Por lo que, como queríamos probar:
	\[\norm{\pi (t) - \pi (t_0) - \int_{t}^{t_0} f(s, \pi(s))ds} \leq \varepsilon |t-t_0|\]
	
	\textbf{Caso III.} Ejercicio propuesto. Basta con realizar inducción y basarse en el caso II.
\end{proof}
\begin{theorem}[Teorema de caracterización de la solución]\label{carac-sol}
	Sea la ecuación diferencial definida por $ f:(t,x) \in A \subset \mathbb{R}\times \mathbb{R}^n\longrightarrow f(t,x)\in \mathbb{R}^n$, tal que $ x'=f(t,x) $. 
	
	Sea $f$ continua y $\varphi :t \in I \subset \mathbb{R} \longrightarrow \mathbb{R}^n$ tal que $(t, \varphi (t)) \in A$ $\forall t \in A$. Equivalen:
	\begin{center}
		$\varphi$ solución de la ecuación $\Leftrightarrow \begin{cases}
		(1) \hspace{0,2cm} \varphi \hspace{0,2cm} continua \\
		(2) \hspace{0,2cm} \varphi (t)= \varphi (t_0) + \int_{t_0}^{t} f(s, \pi (s))ds
		\end{cases}$
	\end{center}
\end{theorem}
\begin{proof}\ \\
	$"\Rightarrow"$ \\
	Supongamos $\varphi$ solución de $x' = f(t, x)$ tenemos entonces que:
	\begin{enumerate}[\qquad i)]
	\item $(t, \varphi (t)) \in A$ $\forall t \in I$.
	\item $ \exists \varphi' (t)$ $\forall t \in I$.
	\item $ \varphi' (t) = f(t, \varphi (t)) t \in I$.
	\end{enumerate}
	En primer lugar $ii)$ implica que $\varphi$ continua. Además, por $iii)$ sabemos que $\varphi' (t) = f(t, \varphi (t))$. Vemos que la composición:
	\[t \stackrel{g}{\longrightarrow } (t, \varphi (t)) \stackrel{f}{\longrightarrow } f(t, \varphi (t)) = \varphi'(t)\]
	es continua por ser continuas $g$ y $f$, por lo que $\varphi '$ es integrable. Entonces, por el Teorema Fundamental del cálculo,
	\[\varphi (t) - \varphi (t_0) = \int_{t_0}^{t} \varphi' (s)ds = \int_{t_0}^{t} f(s, \varphi (s))ds \Rightarrow \varphi (t) = \varphi (t_0) + \int_{t_0}^{t} f(s, \varphi (s))ds\]
	$"\Leftarrow"$ \\
	Ahora supongamos que se cumplen $(1)$ y $(2)$. Por hipótesis $(t, \varphi (t)) \in A$ $\forall t \in I$. Pero, ¿$\exists \varphi' (t)$ $\forall t \in I$?\\
	Tenemos en primer lugar que $f$ continua y que $\varphi$ es continua, lo que implica que $\int_{t_0}^{t} f(s, \varphi (s))ds$ es derivable y su derivada vale $f(t, \varphi (t))$. Por tanto $\varphi (t)$ es suma de una constante y una función derivable con derivada $ f(t,\varphi(t)) $. \\
	Es decir, es derivable y $\varphi' (t) = f(t, \varphi (t))$.
\end{proof}
\subsection{Existencia de la solución. Teorema de Cauchy-Peano.}
\begin{theorem}\label{thm:sucesion}
	Sea $\{\varepsilon_m\}$ sucesión de números reales $\varepsilon_m \geq 0$ tal que $\{\varepsilon_m\}_{m\in \mathbb{N}}\xrightarrow{m\to \infty}0$ y sea $I \subset \mathbb{R}$ intervalo compacto. 
	Sea la ecuación diferencial habitual, $ f:(t,x) \in A \subset \mathbb{R}\times \mathbb{R}^n\longrightarrow f(t,x)\in \mathbb{R}^n$, tal que $ x'=f(t,x) $, con $A$ abierto y $f$ continua en $A$.
	
	Supongamos que $\exists \pi_m : I \longrightarrow \mathbb{R}^n$ sucesión de soluciones $\varepsilon_m$-aproximadas de la ecuación diferencial. Entonces, si $\exists \ \varphi : t \in I \longrightarrow \mathbb{R}^n$ tal que $(t, \varphi (t)) \in A$ $\forall t \in I$ y $\{\pi_m\}_{m\in \mathbb{N}}$ converge a $\varphi$ uniformemente en $I$, entonces $\varphi$ es solución de la ecuación diferencial. 
\end{theorem}

\begin{definition}
	Sea $I \subset \mathbb{R}$ intervalo compacto, $M \subset C(I, \mathbb{R})$ es \emph{equicontinuo}\footnote{ $ C(I, \mathbb{R})$ denota el conjunto de funciones continuas de $ I $ en $ \mathbb{R} $.} si:
	\[\forall \varepsilon > 0 \hspace{0,2cm} \exists \delta > 0 \hspace{0,2cm} / \hspace{0,2cm} \forall x, y\in |x-y| < \delta \Rightarrow \norm{h(x) - h(y)} < \varepsilon, \forall \ h \in M.\]
	Intuitivamente, significa que todas las funciones de un conjunto sean continuas y varíen de manera parecida sobre entornos del dominio.
\end{definition}
\begin{example} \ \\
	Ejemplos de conjuntos de funciones equicontinuos son:
	\begin{enumerate}
		\item Un conjunto finito de funciones de $C(I, \mathbb{R}^n)$. Tomemos $n = 2$. Sea $\varepsilon_0 > 0$ se tiene que:
		\begin{center}
			$\begin{cases}
			h_1 \hspace{0,2cm} continua \hspace{0,2cm} \exists \delta_1 > 0 \hspace{0,2cm} / \hspace{0,2cm} |x-y|<\delta_1 \Rightarrow \norm{h_1(x) - h_1(y)} < \varepsilon_0 \\
			h_2 \hspace{0,2cm} continua \hspace{0,2cm} \exists \delta_2 > 0 \hspace{0,2cm} / \hspace{0,2cm} |x-y|<\delta_2 \Rightarrow \norm{h_2(x) - h_2(y)} < \varepsilon_0
			\end{cases}$
		\end{center}
		Basta tomar $\delta = min\{\delta_1, \delta_2\}$.
		\item Un conjunto de funciones lipschitzianas con la misma constante de Lipschitz $k$.
		\begin{center}
			$\norm{h(x) - h(y)}\leq k|x-y|$ $\forall h \in M$ \\
			$\varepsilon > 0$ $\exists \delta > 0$ tal que $|x-y|<\delta \Rightarrow \norm{h(x) - h(y)} < k\delta = \varepsilon$
		\end{center}
	\end{enumerate}
\end{example}

BEGIN MANU 13/2 

\begin{theorem}[Construccion de solución $ \varepsilon $-aproximada.]
	Consideramos la ecuación diferencial definida por $ f:(t,x) \in A \subset \mathbb{R}\times \mathbb{R}^n\longrightarrow f(t,x)\in \mathbb{R}^n$, tal que $ x'=f(t,x) $, con $ A $ abierto, $ f $ continua y acotada en $ A $ y $ \varepsilon\in \mathbb{R}, \varepsilon>0$. 
	
	Entonces fijado $ (t_0, x_0) \in A $ existe solución $ \varepsilon $-aproximada pasando por $ (t_0, x_0)$.
\end{theorem}
\begin{proof} \ 
	
	\begin{figure}[h]
		\centering
	\begin{tikzpicture}
	\draw[->] (-1,0) -- (5,0) node[right] {$t \in \mathbb{R}$}; %eje x, en este caso t
	\draw[->] (0,-0.5) -- (0,3.5) node[above] {$x \in \mathbb{R}^n$};% eje y
	\draw  plot[tension=.7] coordinates {(1,1) (1,2.5) (4,2.5) (4,1) (1,1) }; %conjunto R
	\draw  plot[scale=0.53,shift={(4.3,2.75)},smooth, tension=.7] coordinates {(-3.5,0.5) (-3,2.5) (-1,3.5) (1.5,3) (4,3.5) (5,2.5) (5,0.5) (2.5,-2) (0.5,-1) (-3,-2) (-3.5,0.5)}; % conjunto A
	\draw[-] (2.5,0.10) -- (2.5,-0.10) node[below][scale=0.8]{$t_0$}; %t0 y puntos de x
	\draw[-] (1,0.10) -- (1,-0.10) node[below][scale=0.8]{$t_0-\delta$};
	\draw[-] (4,0.10) -- (4,-0.10) node[below][scale=0.8]{$t_0+\delta$};
	\draw[-] (3.0,0.10) -- (3.0,-0.10) node[below][scale=0.8]{$t_1$};
	\draw[-] (-0.10,1.75) -- (0.10,1.75) node[left=2mm][scale=0.8]{$x_0$}; %x_0 y puntos de y
	\draw[-] (-0.10,1) -- (0.10,1) node[left=2mm][scale=0.8]{$x_0-H\delta$};
	\draw[-] (-0.10,2.5) -- (0.10,2.5) node[left=2mm][scale=0.8]{$x_0+H\delta$};
	\coordinate[label=above:A] (A) at (5.2,2.8); %etiqueta de A
	\coordinate[label=above:R] (R) at (4.2,2.2); %etiqueta de R
	\draw  plot[tension=.7] coordinates {(1.25,1.75) (1.5,2.2) (1.7,2) (2.0,2.4) (2.5,1.75) (3.0, 2.3) (3.4, 1.9) (3.7, 2.3)}; %linea poligonal
	\fill (2.5,1.75)  circle[radius=1pt]; %punto en t0,x_0
	\coordinate[label={[scale=0.6] right:$(t_0,x_0)$}] (p0) at (2.5,1.7); %etiqueta de t0 x0
	\coordinate[label={[scale=0.6] above:$\rho_1$}] (rho1) at (2.7,2.0); %etiqueta de rho 1
	\draw[-,dashed] (2.5,0.9275) -- (2.5, 1.75) node[right=1mm][below=3mm][scale=0.6]{$d$}; %linea punteada de distancia
	\end{tikzpicture}
	\caption{Esquema orientativo de la demostración.} \label{M1}
\end{figure}
	
	Sea $ d=d_\infty((t_0, x_0), \text{Fr} (A))>0 $ (ya que $ A $ abierto).
	\[ d=\inf \{d_\infty ( (t_0,x_0), (t,x)):(t,x)\in \text{Fr} (A) \}=\inf \{\max \{|t-t_0|,||x-x_0||\}:(t,x)\in \text{Fr} (A)\}\]
	Como $ f $ acotada, $ \exists H $ para $ f $, tal que $ \forall(t,x) \in A, ||f(t,x)||<H $.
	
	Definamos el siguiente conjunto $ R $, con $ \delta $ tal que $ 0<\delta<\min \{d, \frac{d}{H}\} $:
	\[ R=\{(t,x):|t-t_0|<\delta, ||x-x_0||\leq H\delta\} \]
	Veamos que $ R \subset A $. $ (t_1,x_1)\in R \Rightarrow \begin{cases}
	|t_1-t_0|<\delta<\min\{d,\frac{d}{H}\}\leq d\Rightarrow |t_1-t_0|<d\\
	||x_1-x_0||\leq H\delta<H\min\{d,\frac{d}{H}\}\leq H \frac{d}{H}=d \Rightarrow ||x_1-x_0||<d
	\end{cases}$  
	
	$\begin{rcases}
	\text{De estas dos condiciones se deduce } d_\infty ( (t_0,x_0), (t_1,x_1)) <d\\
	A \text{ abierto}\\
	d=d_\infty((t_0, x_0), \text{Fr} (A))
	\end{rcases} \Rightarrow (t_1,x_1)\in A \Rightarrow R \subset A$.
	
	Si existe $ \varphi $ solución de la ecuación diferencial del enunciado, la tangente a $ \varphi $ en $ (t_0,\varphi(t_0))=(t_0,x_0) $ sería (ya que $ \varphi $ es solución):
	\[ \varphi'(t)|_{t_0}=f(t, \varphi(t))|_{t_0}=f(t_0,\varphi (t_0))=f(t_0,x_0) \]
	Definimos entonces la siguiente función, que aproxima mediante una recta a $ \varphi $ en un entorno de $ (t_0,x_0) $: 
	\[ \rho_1:I\subset \mathbb{R}\longrightarrow \mathbb{R}^n \]
	\[ t \mapsto \rho_1(t)=x_0+f(t_0,x_0)(t-t_0) \]
	Es la ecuación de la recta pasando por $ (t_0,x_0) $ con pendiente $ f(t_0,x_0) $.
	
	Veamos para que valores de $ t $, $ \rho_1(t) $ es solución $ \varepsilon $-aproximada. 
	Si $ t\geq t_0 $, la pendiente de $ \rho_1 $ en $ t $ es $ f(t_0,x_0) $.
	Como $ f $ acotada en $ A $, tenemos $ ||f(t_0,x_0)||\leq H $.
	
	Veamos si $ \rho_1 $ verifica las condiciones de ser $ \varepsilon $-aproximada.
	\begin{enumerate}[\qquad i)]
		\item  $ (t,\rho(t)) \in \mathbb{R}\subset A$.
		
		\item $ \rho_1 $ es continua y $ C^n $ a trozos ya que $ \rho_1 \in C^\infty $.
		
		\item ¿Se cumple $ ||\rho_1'(t)-f(t,\rho_1(t))||\leq \varepsilon $?
	\end{enumerate}
	
	Por una parte, $ ||\rho_1'(t)-f(t,\rho_1(t))||=||f(t_0,x_0)-f(t,x_0+f(t_0,x_0)(t-t_0))||  $
	
	Por otra, $ \begin{rcases}
	f \text{ continua en } A\\
	R\subset A \text{ compacto}
	\end{rcases} \Rightarrow f \text{ uniformemente continua en } R\text{, es decir, } \forall \varepsilon > 0 \ \exists \hat{\delta} >0 : \forall y,z \in R, ||y-z||<\hat{\delta} \Rightarrow||f(y)-f(z)||<\varepsilon $.
	
	Tenemos que demostrar que $ ||\rho_1'(t)-f(t,\rho_1(t))||=||f(t_0,x_0)-f(t,x_0)+f(t_0,x_0)(t-t_0)||=||f(y)-f(z)||<\varepsilon $. 
	
	Por lo anterior ($ f $ uniformemente continua), si para $ \varepsilon>0 $ probamos que $||y-z||=||(t_0,x_0)-(t, x_0+f(t_0,x_0)(t-t_0))||< \hat{\delta} $, ya queda demostrado que $ ||f(y)-f(z)||<\varepsilon \Leftrightarrow iii) $.
	
	Veamos para que valores de $ t $, se tiene $ ||(t_0,x_0)-(t, x_0+f(t_0,x_0)(t-t_0))||< \hat{\delta} \Rightarrow ||(t_0-t),( x_0-x_0 -f(t_0,x_0)(t-t_0))||=||(t_0-t),(-f(t_0,x_0)(t-t_0))||< \hat{\delta} $.
	
	Notemos que al trabajar con la norma infinito lo que tenemos que comprobar es que los máximos de los valores absolutos de las componentes son menores que la cota, es decir:	
	\[ \begin{rcases}
	 |t-t_0|< \hat{\delta}\\
	 \text{ y }\qquad\\
	|| -f(t_0,x_0)(t-t_0)||< \hat{\delta}
	\end{rcases} \Rightarrow ||(t_0-t), f(t_0,x_0)(t-t_0)||< \hat{\delta} \]
	Sea $ \alpha >0 $ tal que $ \alpha < \min \{ \hat{\delta}, \frac{\hat{\delta}}{H}\} $ veamos que $ \forall t \in [t_0,t_0+\alpha] $, $\rho_1(t)$ es solución $ \varepsilon $-aproximada, es decir, se cumple $ ||t_0-t,f(t_0,x_0)(t-t_0)||< \hat{\delta}\Rightarrow ||\rho_1'(t)-f(t,\rho_1(t))||\leq \varepsilon $.
	
	\[ t \in [t_0, t_0+\alpha]\Rightarrow |t-t_0|\leq \alpha  
	\begin{rcases}
	< \hat{\delta}\leq\frac{\hat{\delta}}{H} \text{(si } min\{ \hat{\delta}, \frac{\hat{\delta}}{H}\}= \hat{\delta})\ \  \\
	< \frac{\hat{\delta}}{H}<\hat{\delta} \text{(si }  min\{ \hat{\delta}, \frac{\hat{\delta}}{H}\}=  \frac{\hat{\delta}}{H})
	\end{rcases}\Rightarrow \begin{cases}
	|t-t_0|<\hat{\delta}\\
	|t-t_0|<\frac{\hat{\delta}}{H}
	\end{cases}
	 \]
	 Solo falta comprobar si $ || -f(t_0,x_0)(t-t_0)||< \hat{\delta} $:	 
	  \[ t \in [t_0, t_0+\alpha]\Rightarrow ||-f(t_0,x_0)(t-t_0)|| = || f(t_0,x_0)||\cdot|t-t_0|<H\alpha < H \frac{\hat{\delta}}{H}= \hat{\delta}   \]
	Entonces, tenemos visto que $ \rho_1 $ es solución $ \varepsilon $-aproximada en $ [t_0,t_0+\alpha] $. Sea ahora $ t_1=t_0+\alpha $ y $ x_1=\rho_1(t_1) $ y definimos la siguiente función:
	\[ \rho_2:I\longrightarrow \mathbb{R} \]
	\[ t \longmapsto \rho_2(t)=x_1+f(t_1,x_1)(t-t_1) \]
	veremos de forma análoga que $ \rho_2(t) $ es solución $ \varepsilon $-aproximada $  \forall t \in [t_1,t_1+\alpha]  $, repitiendo el proceso para construir la solución $ \varepsilon $-aproximada por poligonales. 
	
	BEGIN PRADA 18/02
	
	Así, $ \varphi_2 $ es solución $ \forall t \in [t_1,t_1+\alpha] $, si $ t_1+\alpha\leq t_0+\delta $ o bien $\forall t \in [t_1, t_0+\delta] $ si $ t_1+\alpha >t_0+\delta $. 
	
	Si $ t_1+\alpha \leq t_0 +\delta $, seguimos construyendo hasta encontrar un $ k $ tal que $ \begin{cases}
	t_0+\alpha (k-1)<t_0+\delta\\
	t_0+\alpha k\geq t_0 +\delta
	\end{cases} $.
	
	Llamando $ t_i=t_0+\alpha i $,
	
	\begin{adjustwidth}{1cm}{}
		$ \rho_1(t)=x_0+f(t_0,x_0)(t-t_0) $
	
		$ \rho_2(t)=x_1+f(t_1,x_1)(t-t_1)=\rho_1(t_1)+f(t_1,x_1)(t-t_1) $
	
		$ \qquad \vdots $
	
		$\rho_i (t)=\rho_{i-1}(t_{i-1})+f(t_{i-1},x_{i-1})(t-t_{i-1})  $, con $ x_{i-1}=\rho_{i-1}(t_{i-1}) $
	\end{adjustwidth}

	Entonces $ \pi: t \in [t_0, t_0+\delta]\longrightarrow \pi(t)=\rho_i(t), t \in [t_{i-1}, t_i], i=1\dots k  $ es solución $ \varepsilon $-aproximada de la ecuación diferencial del enunciado en $ [t_0, t_0+\delta] $. Falta el caso $ t\leq t_0 $ para en $ [t_0-\delta, t_0] $, que es análogo.
\end{proof}

\begin{exercise}
	La solución  $ \varepsilon $-aproximada construida verifica:
	
	1) $ ||\pi(t)||\leq |x_0|+H\delta \ \forall t \in [t_0-\delta,t_0+\delta]$
	
	2) $ \pi $ es lipschitziana con constante $ H $, i.e., $ ||\pi(t)- \pi(s)|| \leq H |t-s| \ \forall t,s \in [t_0-\delta, t_0+\delta] $
\end{exercise}

\begin{note}
	La solución $ \varepsilon $-aproximada construida es local en $ (t_0,x_0) $, es decir, solamente está definida en un entorno del punto.
\end{note}

\begin{definition}
	Diremos que un conjunto $ M $ es \emph{relativamente compacto} si verifica: toda sucesión de elementos de de $ M $ posee una subsucesión convergente.
\end{definition}
\begin{theorem}[Ascoli-Arzelá]
	Sea $ I\subset \mathbb{R} $ intervalo compacto, $ M \subset \mathcal{C} (I, \mathbb{R}^n) $.
	
	$ M $ es relativamente compacto si y solo si:
	\begin{enumerate}[\quad 1)]
		\item $ M $ es equicontinuo.
		\item  $ M $ es puntualmente acotado ($ \forall t \in I,\exists K $ cte $ :||h(t)||\leq K , \forall h \in M$)
	\end{enumerate}
\end{theorem}
\begin{theorem}[Cauchy-Peano]
	Sean $ A $ un abierto de $ \mathbb{R}^{n+1} $ y $ f $ la aplicación $ f:(t,x)\in A\subset \mathbb{R}\times \mathbb{R}^n \longrightarrow f(t,x)\in \mathbb{R}^n $, con $  x'=f(t,x) $ su ecuación diferencial asociada. 
	
	Fijado un punto $ (t_0,x_0) \in A $, si $ f $ es continua en $ A $, existe solución de la ecuación pasando por $ (t_0,x_0) $, definida en un entorno de $ t_0 $.
\end{theorem}
\begin{proof} \ \\
	\begin{figure}[h]
		\centering
		\begin{tikzpicture}
		\draw  plot[tension=.7] coordinates {(1.5,1.5) (1.5,2.5) (3,2.5) (3,1.5) (1.5,1.5) }; %conjunto R
		\draw  plot[scale=0.53,shift={(4.3,3)},smooth, tension=.7] coordinates {(-3.5,0.5) (-3,2.5) (-1,3.5) (1.5,3) (4,3.5) (5,2.5) (5,0.5) (2.5,-2) (0.5,-1.5) (-3,-2) (-3.5,0.5)}; % conjunto A
		\coordinate[label=above:A] (A) at (5.2,2.8); %etiqueta de A
		\coordinate[label=above:R] (R) at (2.7,2.4); %etiqueta de R
		\coordinate[label=above:K] (K) at (3.3,2.2); %etiqueta de K
		\fill (2.0,2)  circle[radius=1pt]; %punto en t0,x_0
		\coordinate[label={[scale=0.6] right:$(t_0,x_0)$}] (p0) at (2,2); %etiqueta de t0 x0
		\draw (2,2) circle (1.2);
		\end{tikzpicture}
		\caption{Esquema de la demostración.} \label{M2}
	\end{figure}
	Sea $ \{\varepsilon_m \} $ sucesión de números reales, $ \varepsilon_m>0  $ tal que $ \lim\limits_{m\to\infty} \{\varepsilon_m\} =0$.
	
	Sea $ K $ compacto ($ K $ bola de centro $ (t_0,x_0) $ y radio $ r<d_\infty((t_0,x_0), \text{Fr}(A)), K\subset A, (t_0,x_0) \in \mathring{K} $).
	
	$ \begin{rcases}
	K \text{ compacto}\\
	f \text{ continua en } K
	\end{rcases} \Rightarrow  f $ acotada en $ K $.
	
	Aplicando el teorema de construcción de solución $ \varepsilon $-aproximada, $ \exists I=[t_0-\delta, t_0+\delta] $ en el cual esta definida $ \pi_m $ solución  $ \varepsilon $-aproximada para cada $ \varepsilon_m $.
	
	Por el ejercicio, $ \begin{cases}
		||\pi_m(t)||< x_0+H\delta \ \forall m \text{ con } H \text{ cota para } f.\ (1)\\
		\pi_m \text{ es } H\text{-lipschitziana.}
	\end{cases} $
	
	$ M = \{\pi_m\} \subset  \mathcal{C} (I, \mathbb{R}^n)$ es equicontinuo porque todas sus elementos son funciones lipschitzianas con la misma constante $ H $.
	
	$ M $ es puntualmente acotado por (1). 
	
	Como $ M $ es puntualmente acotado y equicontinuo, por el Teorema de Ascoli-Arzelá, $ M$ es relativamente compacto $ \Rightarrow \ \exists \{\pi_{m_k}\}$ subsucesión de $ \{\pi_m\} $ que converge a una función $ \varphi\in \mathcal{C}(I, \mathbb{R}^n) $.
	
	$ \begin{rcases}
		\text{Por tanto } \exists \varphi: I\longrightarrow \mathbb{R}^n \text{ tal que } \{\pi_{m_k} \}\longrightarrow \varphi \text{ uniformemente }\\
		\{\varepsilon_{m_k}\xrightarrow{k\to \infty}0  \} \text{ por ser subsucesión.}
	\end{rcases} \Rightarrow \varphi$ solución de $ x'=f(t,x) $, por el Teorema \ref{thm:sucesion}.
	
	Por construcción, $ \pi_{m_k}(t_0)=x_0 \Rightarrow \varphi(t_0) = \lim\limits_{k\to \infty}\pi_{m_k}(t_0)=x_0$.
\end{proof}
BEGIN PRADA 18/02

BEGIN JORGE 20/02
\subsection{Unicidad de la solución. Teorema de Picard-Lipschitz.}
\begin{example}
	El siguiente ejemplo es de una ecuación con varias soluciones. Es la ecuación:
	\[x'=x^{\frac{2}{3}}\]
	Así, tenemos que:
	\[f:(t, x)\in A = \mathbb{R} \times \mathbb{R} \longrightarrow f(t, x) = x^{\frac{2}{3}} \in \mathbb{R}\]
	y estas son posibles soluciones: 
	\begin{enumerate}
		\item Tenemos que la aplicación:
		\begin{align*}
		\varphi_1: t \in I \subset \mathbb{R}& \longrightarrow \mathbb{R}\\
		t& \longrightarrow \varphi_1 (t) = 0
		\end{align*}
		es solución pasando por el $(0,0)$, ya que:\\
		$
		\begin{rcases}
			\quad&\text{i) } (t, \varphi_1 (t)) \in A \\
			&\text{ii) }\exists \varphi_1' (t),  \hspace{0,2cm} \forall t \in I \\
			&\text{iii) } \varphi_1' = f(t, \varphi_1 (t)), \hspace{0,2cm} \forall t \in I
		\end{rcases}$ $\Rightarrow \varphi_1$ solución de $x'=x^{\frac{2}{3}}$ pasando por $(0,0)$
		
		\item También se tiene que $\varphi_2: t \in \mathbb{R} \longrightarrow \varphi_2 (t) = (\frac{t}{3})^3$ es solución, ya que:\\
		$
		\begin{rcases}
			\quad&\text{i) } (t, \varphi_2 (t)) \in A \\
			&\text{ii) }\exists \varphi_2' (t),  \hspace{0,2cm} \forall t \in I \\
			&\text{iii) } \varphi_2' = f(t, \varphi_2 (t)), \hspace{0,2cm} \forall t \in I
		\end{rcases}$
		\item Por el mismo motivo también son solución:\\
		$\varphi_3 (t) = \begin{cases}
			0, &t\leq 0 \\
			(\frac{t}{3})^3, &t>0 
		\end{cases}$ \\
		$\varphi_4 (t) = \begin{cases}
		(\frac{t}{3})^3, &t\leq 0\\
		0, &t>0 
		\end{cases}$ 
	\end{enumerate}
EL teorema de Cauchy-Peano no asegura nada respecto a la unicidad de la solución.
\end{example}
\begin{definition}
	Sea la ecuación diferencial definida por  $ f:(t,x) \in A \subset \mathbb{R}\times \mathbb{R}^n\longrightarrow f(t,x)\in \mathbb{R}^n$, tal que $ x'=f(t,x) $. 
	
	Decimos que tiene \emph{solución única} pasando por $(t_0, x_0) \in A$ si cualesquiera que sean $\varphi_1 : I_1 \longrightarrow \mathbb{R}^n$ y $\varphi_2 : I_2 \longrightarrow \mathbb{R}^n$ soluciones pasando por $(t_0, x_0)$ se cumple que $\varphi_1 (t) = \varphi_2 (t)$ $\forall t \in I_1 \cap I_2$.
\end{definition}
\begin{definition}
	Sea $ f:(t,x) \in A \subset \mathbb{R}\times \mathbb{R}^n\longrightarrow f(t,x)\in \mathbb{R}^n$.
	
	Decimos que $f$ es \emph{lipschitziana} en $A$ con respecto a la variable $x$ si $\exists k \in \mathbb{R}$, $k \geq 0$ tal que:
	\[\norm{f(t, x) - f(t, y)} \leq k\norm{x-y} \hspace{0,2cm} \forall (t, x), (t, y) \in A\]
	Se denota $f\in L(A, x)$ o bien $f \in L_k (A, x)$. Conviene recordar que una función lipschitziana es uniformemente continua y por tanto continua, que si $ k=1 $ la función se denomina función corta y si $ k<1 $ es una función contractiva.
\end{definition}
\begin{example}
	Sea $f(t, x) = x^2$, ¿es $f$ lipschitziana con respecto a $x$? Será lipschitziana si $\norm{f(t, x) - f(t, y)} < k\norm{x-y}$. ¿Existe tal $k$? \\
	Si $x \neq y$, 
	\[\frac{\norm{x^2 - y^2}}{x-y} \stackrel{??}{\leq} k\]
	\[\frac{\norm{x^2 - y^2}}{x-y} \leq \frac{\abs{x^2 - y^2}}{\abs{x - y}} = \abs{\frac{x^2 - y^2}{x - y}} = \abs{x+y} \leq k\]
	Por lo que no existe $k$ tal que $\abs{x+y} \leq k$, luego $f$ no es lipschitziana.
\end{example}
\begin{note}
	Una de las condiciones que pedimos para que $f$ sea lipschitziana es que:
	\[\frac{\norm{f(t, x) - f(t, y)}}{\norm{x - y}}\]
	esté acotado. 
\end{note}
\begin{definition}
	Sea $f: A \subset \mathbb{R} \times \mathbb{R}^n \longrightarrow \mathbb{R}^n$. Decimos que $f$ es \emph{localmente lipschitziana} en $A$ con respecto a la variable $x$ si $\forall (t_0, x_0) \in A$, $\exists\xi_0$ entorno de $(t_0, x_0)$ tal que $f \in L(\xi_0, x)$. Se denota como $f \in L_{loc}(A, x)$.
\end{definition}
\begin{theorem}
	Sea $A = I \times V$, con $I$ intervalo e $V$ convexo. Se tiene:
	\begin{enumerate}
		\item Si $\exists D_x f$ y está acotada $\Rightarrow f \in L(A, x)$
		\item Si $\exists D_x f$ y es continua $\Rightarrow f \in L_{loc}(A, x)$
	\end{enumerate}
\end{theorem}
\begin{proof}\ 
	\begin{enumerate}
		\item Por el teorema de incrementos finitos:
		\[\norm{f(t, x_1) - f(t, x_2)} \leq \norm{D_xf(t, x_3)} \norm{x_2 - x_1} \leq k\norm{x_2 - x_1}\]
		\item Al no estar acotada, no podemos afirmar lo mismo que en 1 para todo el dominio. Sin embargo, en todo entorno compacto, una función continúa si está acotada y en ese entorno podemos proceder igual que en el punto 1, asegurando lipschitzianidad local.
	\end{enumerate}
\end{proof}
\begin{definition}
	Sea $(M, d)$ un espacio métrico y sea $T: M \longrightarrow M$, $0 \leq \lambda < 1$. Se dice que $T$ es una \emph{$\lambda$-contracción} si $\forall x, y \in M$, 
	\[d(T(x), T(y)) \leq k \, d(x, y) \stackrel{k < 1}{\Leftrightarrow} d(T(x), T(y)) < d(x, y)\]
\end{definition}
\begin{observation}
	Las funciones en el espacio euclídeo usual lipschitzianas contractivas ($ k<0 $) son $\lambda$-contracciones con $ \lambda=k $.
\end{observation}
\begin{theorem}[Teorema de la transformada contractiva]\label{transformada}
	Sea $(M, d)$ un espacio métrico completo y $T: M \longrightarrow M$ una $\lambda$-contracción entonces $T$ tiene un único punto fijo, es decir,
	\[\exists! \, x \in M \,/\, T(x) = x\]
\end{theorem}
\begin{theorem}[Teorema de Picard - Lipschitz]
	Sea $f : A \subset \mathbb{R} \times \mathbb{R}^n \longrightarrow \mathbb{R}^n$ y $x' = f(t, x)$ su ecuación diferencial asociada. Si:
	\begin{enumerate}[\quad i)]
		\item $A$ abierto.
		\item $f \in C(A)$.
		\item $f \in L_{loc} (A, x)$.
	\end{enumerate}
	entonces, fijado $(t_0, x_0) \in A$, existe solución pasando por $(t_0, x_0) \in A$ y es única.
\end{theorem}
\begin{note}
	Como en $f(t, x) = x' = x^{\frac{2}{3}}$ tenemos que es continua y no tiene solución única, entonces $f \notin L_{loc}(\mathbb{R}^2,x)$. \\
	En $(t, 0)$ $f \notin L_{loc}(\mathbb{R}^2,x)$ pues si $\xi$ es un entorno de $(t, 0)$, tomando $(t, x), \, (t, y) \in \xi$, con $y=0$:
	\[\frac{\abs{f(t, x) - f(t, y)}}{\abs{x - y}} = \abs{\frac{x^{\frac{2}{3}}}{x}} = \abs{x^{-\frac{1}{3}}}\]
	y como:
	\[\lim\limits_{x \to 0} \frac{1}{x^{\frac{1}{3}}} = +\infty\] 
	vemos que en efecto, no es localmente lipschitziana.
\end{note}
	BEGIN JORGE 25/02
\begin{proof}\ \\
	La demostración se basa en dos resultados, el teorema \ref{transformada} de la transformada contractiva y el teorema \ref{carac-sol} de caracterización de la solución. En primer lugar definiremos una serie de elementos que se emplearán a lo largo de la demostración:
	\begin{enumerate}
		\item Sea $ B_1 $  entorno abierto de $ (t_0,x_0) $ en el cual $ f $ está acotada, que existe por ser $ f $ continua. Sea $ H $ cota para $ f $.
		
		\item Sea $ B_2 $ un entorno abierto de $ (t_0,x_0) $ en el cual $ f $ es lipschitziana. (Sea $ K $ constante de Lipschitz).
		
		\item Sea $ B=B_1 \cap B_2 $, $ B $ abierto, $ (t_0,x_0) \in B$.
		
		\item Sea $ d=d_\infty((t_0,x_0),\text{Fr}(B)) $ y elegimos $ \delta : 0<\delta <\min \{d,\frac{d}{H},\frac{1}{K}\} $. Añadimos $ \frac{1}{K} $ para probar $ \lambda $-contracción.
		\item Llamamos $ R=\{(t,x) : |t-t_0|\leq \delta, ||x-x_0||\leq H\delta \}\subset B $.
		DIBUJO
	\end{enumerate}
	
	Definimos el conjunto:
	 \[X=\{\varphi : I=[t_0-\delta, t_0+\delta]\longrightarrow \mathbb{R}^n : \varphi \text{ continua, } ||\varphi(t)-x_0||\leq H \delta \hspace{0,2cm} \forall t \in I  \} \]
	de las aplicaciones continuas cuyo grafo está en $R$. Dotaremos a este conjunto $ X $ con la siguiente métrica. Dadas $\varphi, \psi \in X$, $d(\varphi, \psi) = sup_{t \in I} \norm{\varphi (t) - \psi (t)}$ (como ejercicio probar que es una métrica). Así tenemos que:\\
	\[\begin{rcases}
		C(I, \mathbb{R}^n) \text{ espacio métrico completo \footnotemark}\\
		X = B_{C(I, \mathbb{R}^n)} [x_0, H\delta] \subset C(I, \mathbb{R}^n) \text{ cerrado } \\
	\end{rcases} \Rightarrow (X, d) \text{ completo.}\]
	\footnotetext{Un espacio métrico completo es aquel en el que toda sucesión de Cauchy converge a un punto en el mismo conjunto. En la demostración usamos el siguiente resultado: Un subconjunto cerrado de un espacio métrico completo es completo.}
	
	Nótese que $ x_0 $ puede ser entendido como la función constante $ x_0 $, en el contexto de un espacio de funciones. Definimos ahora:
	\[T:\varphi \in X \longrightarrow T\varphi = x_0 + \int_{t_0}^{t} f(s, \varphi(s))ds\]
	Queremos usar el Teorema \ref{transformada} de la transformada contractiva, por tanto, tenemos que demostrar:
	\begin{enumerate}
		\item $T\varphi \in X$.
		\item $T$ es $\lambda$-contracción.
	\end{enumerate}
	Para ver que $T\varphi \in X$ tenemos que ver:
	\begin{enumerate}[	i.]
		\item $T\varphi$ continua
		\item $\norm{T\varphi (t) - x_0} \leq H\delta$, $\forall t \in I$
	\end{enumerate}
	En primer lugar, tenemos que:
	\[s \stackrel{g}{\longrightarrow} (s, \varphi (s)) \stackrel{f}{\longrightarrow} f(s, \varphi (s)) \longrightarrow x_0 + \int_{t_0}^{t}f(s,\varphi (s))ds  \]
	por lo que $T\varphi$ es continua por composición de funciones continuas. \\
	En segundo lugar, 
	\[\norm{T\varphi - x_0} = \norm{x_0 + \int_{t_0}^{t} f(s, \varphi (s)) ds - x_0} = \norm{\int_{t_0}^{t} f(s, \varphi (s)) ds} \leq \]
	\[\abs{\int_{t_0}^{t} \norm{f(s, \varphi (s))}ds} \leq \abs{\int_{t_0}^{t} H ds} = \abs{H(t - t_0)} = H\abs{t - t_0} \leq H\delta \]
	Veamos ahora que $T$ es una $\lambda$-contracción, es decir, $\exists \lambda \in \mathbb{R}$, $0 \leq \lambda < 1$ tal que
	\[d(T\varphi, T\psi) \leq \lambda d(\varphi, \psi) \hspace{0,2cm} \forall \varphi, \psi \in X\]
	Sean $\varphi, \psi \in X$, \\
	\[d(T\varphi, T\psi) = sup_{t \in I} \norm{T\varphi (t) - T\psi (t)} = sup_{t \in I} \norm{x_0 + \int_{t_0}^{t} f(s, \varphi(s))ds - x_0 - \int_{t_0}^{t} f(s, \psi(s))ds} = \]
	\[sup_{t \in I} \norm{\int_{t_0}^{t} (f(s, \varphi(s)) - f(s, \psi(s)))ds} \leq sup_{t \in I} \abs{\int_{t_0}^{t} \norm{(f(s, \varphi(s)) - f(s, \psi(s)))}ds} \stackrel{f \in L_{loc} (A, x)}{\leq} \] 
	\[sup_{t \in I} \abs{\int_{t_0}^{t} k\norm{\varphi(s) - \psi(s)}ds} \leq sup_{t \in I} \abs{\int_{t_0}^{t} k sup\norm{\varphi(s) - \psi(s)}ds} \leq sup_{t \in I} \abs{\int_{t_0}^{t} k d(\varphi, \psi)ds} = \]
	\[sup_{t \in I} kd(\varphi, \psi)\abs{t - t_0} \leq kd(\varphi, \psi)\delta \leq \lambda d(\varphi, \psi)\]
	Tomando $\lambda = \delta k$, como $\delta < \frac{1}{k}$ tenemos que $\lambda < 1$. Nótese que podemos aplicar que $ f $ es $ L_{\text{loc}} $ en $ A $ respecto de $ x $ ya que al ser $ \psi $ y $ \varphi $ funciones de $ X $, están definidas en un entorno de $ t_0 $ en el que $ f $ es lipschitziana, y siempre trabajamos con $ t $ dentro de ese entorno, por como estamos construyendo la prueba.
	
	Por lo tanto, $T$ es $\lambda$-contracción. Es decir, por el Teorema \ref{transformada} de la transformada contractiva:
	\[\exists ! \varphi \in X : \varphi \text{ punto fijo de } T\]
	por lo que $\varphi (t) = T\varphi (t) = x_0 + \int_{t_0}^{t} f(s, \varphi (s))ds$, y como $\varphi$ continua, cumple las dos hipótesis del Teorema \ref{carac-sol} de caracterización de soluciones.	
BEGIN PRADA 27/02

	\begin{figure}[h]
		\centering
		\begin{tikzpicture}
		\draw[->] (-1,0) -- (5,0) node[right] {$t \in \mathbb{R}$}; %eje x, en este caso t
		\draw[->] (0,-0.5) -- (0,3.5) node[above] {$x \in \mathbb{R}^n$};% eje y
		\draw  plot[tension=.7] coordinates {(1,1) (1,2.5) (4,2.5) (4,1) (1,1) }; %conjunto R
		\draw  plot[scale=0.53,shift={(4.3,2.75)},smooth, tension=.7] coordinates {(-3.5,0.5) (-3,2.5) (-1,3.5) (1.5,3) (4,3.5) (5,2.5) (5,0.5) (2.5,-2) (0.5,-1.5) (-3,-2) (-3.5,0.5)}; % conjunto A
		\draw[-] (2.5,0.10) -- (2.5,-0.10) node[below][scale=0.8]{$t_0$}; %t0 y puntos de x
		\draw[-] (1,0.10) -- (1,-0.10) node[below][scale=0.8]{$t_0-\delta$};
		\draw[-] (4,0.10) -- (4,-0.10) node[below][scale=0.8]{$t_0+\delta$};
		\draw[-] (-0.10,1.75) -- (0.10,1.75) node[left=2mm][scale=0.8]{$x_0$}; %x_0 y puntos de y
		\coordinate[label=above:A] (A) at (5.2,2.8); %etiqueta de A
		\coordinate[label=above:R] (R) at (4.2,2.2); %etiqueta de R
		\draw  plot[smooth,tension=.7] coordinates {(1,1.75) (1.5,2.2) (1.7,2) (2.25,2.6) (2.8,0.75) (3.2, 2.6) (3.4, 1.9) (4, 2.3)}; %linea poligonal
		\fill (2.5,1.75)  circle[radius=1pt]; %punto en t0,x_0
		\coordinate[label={[scale=0.6] right:$(t_0,x_0)$}] (p0) at (2.5,1.7); %etiqueta de t0 x0
		\end{tikzpicture}
		\caption{Situación en la que el grafo no está en $ R $.} \label{M1}
	\end{figure}
	¿Podría existir una función que fuera solución pasando por el punto y que el grafo no esté en $ R $? Probemos que no.
	\begin{adjustwidth}{1cm}{}
		Veamos que si $ \psi : I \longrightarrow\mathbb{R}^n $ solución de $ x'=f(t,x) $ pasando por $ (t_0,x_0) $, entonces $ \psi \in X $, es decir, $ ||\psi(t)-x_0||\leq H\delta \ \forall t \in I $.
		
		Probémoslo por reducción al absurdo. Supongamos que $ \exists t_1\in I : ||\psi(t_1)-x_0||>H\delta$, por ser $ \psi $ continua $ \exists t_2\in (t_0,t_1) $ en el cual $ ||\psi(t_2)-x_0||=H\delta $.
		
		Sea $ t^* = \inf \{t : ||\psi(t)-x_0||=H\delta, t>t_0 \} \Rightarrow ||\psi(t^*)-\psi(t_0)||=H\delta$, con $ \psi(t_0)=x_0 $.
		
		Por otra parte, $ ||\psi(t^*)-\psi(t_0)||\leq ||\psi'(t)||\cdot |t^*-t_0|<H\delta $, lo cual es una contradicción.
	\end{adjustwidth}
	
	¿Podría existir otra solución distinta que pasase por el punto?
	\begin{adjustwidth}{1cm}{}
		Veamos que esto tampoco puede pasar. Supongamos que $ \varphi_1: I_1\longrightarrow\mathbb{R}^n, \varphi_2:I_2 \longrightarrow \mathbb{R}^n $ son soluciones de $ x'=f(t,x) $ pasando por $ (t_0,x_0) $ y $ \exists t\in I_1\cap I_2 : \varphi_1(t)\neq \varphi_2(t) $.
		
		Sea por ejemplo $ t>t_0 $. Sea $ t_1=\inf \{t \in I_1\cap I_2 :t>t_0, \varphi_1(t)\neq \varphi_2(t)  \} $.
		
		Por definición de ínfimo $ \varphi_1(t)=\varphi_2(t) \ \forall t \in [t_0, t_1) $ y además $ \varphi_1(t_1)=\lim\limits_{t\to t_1^-} \varphi_1(t) $ por ser $ \varphi_1 $ continua.
		
		$ \varphi_1(t_1)=\lim\limits_{t\to t_1^-} \varphi_1(t)=\lim\limits_{t\to t_1^-} \varphi_2(t)= \varphi_2(t_1) $.
		
		Entonces $ \varphi_1(t)=\varphi_2(t) \ \forall t \in [t_0,t_1], \varphi_1(t_1)=\varphi_2(t_1)=x_1 $.
		
		Hacemos en $ (t_1, x_1) $ lo mismo que en $ (t_0,x_0) $ (construimos el rectángulo, etc...) y llegamos a que en $ (t_1-\delta_1, t_1+\delta _1), \varphi_1(t)=\varphi_2(t) $ y esto NO es posible ya que, por construcción, $ t_1=\inf \{t\in I_1\cap I_2 : t>t_0, \varphi_1(t)\neq \varphi_2(t)\} $.
	\end{adjustwidth}
\end{proof}
\pagebreak
\section{Tema 3.}
BEGIN PRADA 6/3
\begin{example}
	Dada la ecuación diferencial $ x'=x^2 $, pasando por el punto $ x(1)=1 $, tenemos que $ x(t)=\frac{-1}{t} $ es solución. $ x'(t)=x(t)^2 $ es una ecuación planteada para todo $ t \in \mathbb{R} $, pero $ x(t)=\frac{-1}{t} $ está definida para $ t \in \mathbb{R}-\{0\} $.
	
	Entonces tenemos que dependiendo de las condiciones iniciales la solución de la EDO estará definida para $ t\in (0,\infty) $ o $ t\in (-\infty,0) $.
	
	Fijemonos en dos aspectos: con las condiciones iniciales propuestas, $ x(1)=1 $, podemos definir la solución en el intervalo $ (0,3) $. Sin embargo, esta solución podría ser prolongada a un dominio de $ (0,\infty) $, y la posibilidad de esta prolongación será objeto de estudio.
	
	Por otra parte, vemos que al cambiar las condiciones iniciales (por ejemplo, $ x_0=-0.0001 $ o $ x_0=0.00001 $), la solución de la ecuación no es la misma, y en otros caso, podría variar su expresión analítica. Nos interesará que para variaciones "pequeñas" de las condiciones iniciales, la solución de la ecuación no varíe "casi nada".
\end{example}

\begin{note}
	Los teoremas de Cauchy-Peano y Picard-Lipschitz definen existencia y unicidad de soluciones a nivel local, en un entorno arbitrariamente pequeño. Son preguntas naturales: \begin{itemize}
		\item ¿Cuál es el mayor intervalo en el que están definidas esas soluciones?
		\item ¿Hasta donde puedo prolongar las mismas?
	\end{itemize}
\end{note}
\subsection{Prolongación de soluciones. Soluciones maximales.}
\begin{definition}
	Sean $ A $ un abierto de $ \mathbb{R}^{n+1} $ y $ f $ la aplicación continua $ f:(t,x)\in A\subset \mathbb{R}\times \mathbb{R}^n \longrightarrow f(t,x)\in \mathbb{R}^n $, con $  x'=f(t,x) $ su ecuación diferencial asociada. 
	
	Sean además $ \varphi_1:I_1\longrightarrow \mathbb{R} $ y  $ \varphi_2:I_2\longrightarrow \mathbb{R} $ soluciones de dicha ecuación diferencial. Diremos que $ \varphi_2 $ es una \emph{prolongación} de $ \varphi_1 $ si:
	\begin{enumerate}
		\item $ I_1\subseteq I_2 $
		\item $ \varphi_1(t) =\varphi_2(t) \ \forall t \in I_1$
	\end{enumerate}

	Además, si $ I_1 \varsubsetneq I_2 $ diremos que $ \varphi_2 $ es una \emph{prolongación propia} de $ \varphi_1 $.
\end{definition}
\begin{note}
	Una solución es prolongación de si misma, pero no es prolongación propia.
\end{note}
\begin{definition}
	Si $ \varphi $ es solución de una ecuación diferencial, diremos que $ \varphi $ es una \emph{solución maximal} si no admite prolongación propia. En este caso, el intervalo de definición es llamado \emph{intervalo maximal}.
\end{definition}
\begin{lemma}[Lema de Zorn]
	Todo conjunto parcialmente ordenado no vacío, en el que toda cadena (subconjunto parcialmente ordenado) admite cota superior, contiene elementos maximales.
\end{lemma}
\begin{theorem}
	Sean $ A $ un abierto de $ \mathbb{R}^{n+1} $ y $ f $ la aplicación continua $ f:(t,x)\in A\subset \mathbb{R}\times \mathbb{R}^n \longrightarrow f(t,x)\in \mathbb{R}^n $, con $  x'=f(t,x) $ su ecuación diferencial asociada. 
	
	Dado $ (t_0,x_0)\in A $, existe una solución maximal de la ecuación diferencial que pasa por $ (t_0,x_0) $.
	
\end{theorem}
\begin{proof} \ \\	
	Haremos uso del lema de Zorn. Representaremos las soluciones de la ecuación diferencial mediante el par $ (I,\varphi) $ donde $ \varphi $ es la solución e $ I $ su intervalo de definición.
	
	En este conjunto definimos la relación de orden parcial "ser prolongación". Si $ \psi $ es prolongación de $ \varphi $:
	\[ (I,\varphi)\leq (J,\psi)\Leftrightarrow I\subseteq J, \varphi(t)=\psi(t) \ \forall t \in I \]
	Ahora, sea $ (I_0,\varphi_0) $ solución de $ x'=f(t,x) $ y veamos que se puede prolongar hasta una solución maximal.
	
	Trabajamos solo con soluciones que sean prolongación de $ \varphi_0 $. Es un conjunto no vacío, ya que $ (I_0,\varphi_0) $ es prolongación de si mismo. Utilizamos la relación de orden parcial definida anteriormente y comprobamos que se cumple la otra hipótesis del lema de Zorn:
	
	Sea $ \{(I_j,\varphi_j) \}_{j\in I} $ una cadena de ese conjunto y veamos que admite cota superior. Sea $ \varphi $ la función definida en $ I=\bigcup_{j\in I}I_j $ mediante $ \varphi(t)=\varphi_j(t) \forall j \in I $. $ \varphi $ es claramente cota superior de la cadena.
	
	Por tanto, por el lema de Zorn, existen elementos maximales que son las soluciones maximales que prolongan a $ \varphi_0 $.	
\end{proof}
\begin{definition}
	Sean $ A $ un abierto de $ \mathbb{R}^{n+1} $ y $ f $ la aplicación continua $ f:(t,x)\in A\subset \mathbb{R}\times \mathbb{R}^n \longrightarrow f(t,x)\in \mathbb{R}^n $, con $  x'=f(t,x) $ su ecuación diferencial asociada. 
	
	Sean $ \varphi_1:(a,b)\longrightarrow \mathbb{R}^n $ y $ \varphi_2:(c,d)\longrightarrow \mathbb{R}^n $ soluciones de dicha ecuación diferencial. Si $ \varphi_2 $ es una prolongación de $ \varphi_1 $ y $ b\in(c,d) $, diremos que $ \varphi_2 $ es \emph{prolongación de $ \varphi_1 $ a través de $ b $}.
\end{definition}
\begin{definition}[General]
	Sean $ A $ un abierto de $ \mathbb{R}^{n+1} $ y $ f $ la aplicación continua $ f:(t,x)\in A\subset \mathbb{R}\times \mathbb{R}^n \longrightarrow f(t,x)\in \mathbb{R}^n $, con $  x'=f(t,x) $ su ecuación diferencial asociada. 
	
	Sean $ \varphi_1:I_1\longrightarrow \mathbb{R}^n $ y $ \varphi_2:I_2\longrightarrow \mathbb{R}^n $ soluciones de dicha ecuación diferencial. Si $ \varphi_2 $ es una prolongación de $ \varphi_1 $ y existe $ q\in \text{Fr}(I_1) $ tal que $ q\in \mathring{I_2}$, diremos que $ \varphi_2 $ es \emph{prolongación de $ \varphi_1 $ a través de $ q $}.
\end{definition}
\begin{theorem}[Teorema de prolongación]
	Sean $ A $ un abierto de $ \mathbb{R}^{n+1} $ y $ f $ la aplicación continua $ f:(t,x)\in A\subset \mathbb{R}\times \mathbb{R}^n \longrightarrow f(t,x)\in \mathbb{R}^n $, con $  x'=f(t,x) $ su ecuación diferencial asociada. 
	
	Además de estas condiciones habituales, pediremos que $ f\in L_{\text{loc}}(A,x) $, es decir, que $ f $ sea lipschitziana localmente.
	
	Si $ \varphi:(a,b)\longrightarrow\mathbb{R}^n $ es solución de la ecuación diferencial anterior, entonces;
	\begin{enumerate}[\quad i)]
		\item $ \varphi $ prolongable a través de $ b \Leftrightarrow \exists \lim\limits_{t\to b^-}\varphi(t)=p, \text{ con } (b,p) \in A $
		\item $ \varphi $ prolongable a través de $ a \Leftrightarrow \exists \lim\limits_{t\to a^-}\varphi(t)=q, \text{ con } (a,q) \in A $
	\end{enumerate}
\end{theorem}

BEGIN JORGE 11/3

\begin{proof}\ 
	\begin{enumerate} [\quad i)]
		\item "$\Rightarrow$" \\
		Si $\varphi$ prolongable a través de $b$, entonces $\exists \varphi_2 : (a, d) \longrightarrow \mathbb{R}^n$ prolongación de $\varphi$ (es decir, $\varphi_2 = \varphi$ $\forall t \in (a, b)$, $b \in (a, b)$). Por ser $\varphi_2$ solución, $\varphi_2$ es continua por la izquierda en $t = b$, por lo que:
		\[\lim\limits_{t \to b^-} \varphi_2(t) = \varphi_2(b) \text{, } \lim\limits_{t \to b^-} \varphi_2(t) = \lim\limits_{t \to b^-} \varphi(t)\]
		y tenemos entonces que:
		\[\lim\limits_{t \to b^-} \varphi(t) = \varphi_2(t)\]
		Además, $(b, \varphi(b)) \in A$ por ser $\varphi_2$ solución. \\
		"$\Leftarrow$" \\
		Suponemos que:
		\[\exists \lim\limits_{t \to b^-} \varphi(t) = p \text{, } (b, p) \in A\]
		Sea $\overline{B} = B((b, p), r) \subset A$ con $f$ acotada en $\overline{B}$ y $f \in L_k (B, x)$. Tomamos $0 < \delta < min\{\frac{r}{2}, \frac{r/2}{H}, \frac{1}{k}\}$, siendo $H$ cota de $f$.
		
		DIBUJO
		
		Como $\lim\limits_{t \to b^-} \varphi (t) = p$ y, por ser continua, $ \varphi $ es secuencialmente continua, existe una sucesión $\{t_n\} \longrightarrow b^-$ tal que $\{\varphi(t_n) \longrightarrow p\}$. Elegimos entonces un $n_0 \in \mathbb{N}$ tal que $\abs{t_{n_0} - b} < \delta$ y $(t_{n_0}, \varphi (t_{n_0})) \in \overline{B}$. Para la condición inicial $(t_{n_0}, \varphi (t_{n_0}))$ existe una solución única pasando por el punto definido en $[t_{n_0} - \delta, t_{n_0} + \delta]$ que denoto $\overline{\varphi}$. Entonces la función:
		\[\psi: (a, t_{n_0} + \delta] \longrightarrow \psi (t) = \begin{cases}
		\varphi(t) &t \in (a, b) \\ 
		\overline{\varphi} (t) &t \in [b, t_{n_0} + \delta).
		\end{cases}\]
		Así, tenemos que $\psi$ es solución de $x' = f(t. x)$ y, de hecho, $\psi$ prolonga a $\varphi$ a través de $b$.
		
		DIBUJO
		\item Análogo.
	\end{enumerate} 
\end{proof}
\begin{observation}
	En primer lugar, la afirmación $i)$ es equivalente a decir:
	\[\exists \{t_m\} \longrightarrow b^- : \exists \lim\limits_{t_n \to b^-} \varphi(t_n) = p \hspace{0.2cm} (b, p) \in A\]
	Llega trabajar con una sucesión. Además, no es necesario pedir que $f \in L_{loc}(A, x)$, simplemente sin ella no podemos concluir la unicidad y la demostración sería mucho más complicada.
\end{observation}
\begin{corollary}
	Si $A$ está acotado, la solución se puede prolongar hasta la frontera.
\end{corollary}
\begin{theorem}
	(Teorema de la banda) \\
	Dada la ecuación diferencial $x' = f(t, x)$, con $f: (t, x) \in \mathbb{R} \times \mathbb{R}^n \longrightarrow f(t, x) \in \mathbb{R}^n$ siendo $A = (a, b) \times \mathbb{R}^n$. Si se cumple una de las siguientes condiciones:
	\begin{enumerate}
		\item  $f$ continua en $A$, $f \in L_k (A, x)$.
		\item  $f$ continua y acotada en $A$ ($f \in L{loc} (A, x)$).
	\end{enumerate}
	
	DIBUJO
	
	entonces, fijado $(t_0, x_0) \in (a, b) \times \mathbb{R}^n$, la solución maximal que pasa por él, está definida en $(a, b)$.
\end{theorem}
\begin{observation} \ 
	\begin{enumerate}
		\item No es necesario pedir $f \in L_{loc} (A, x)$.
		\item $a$, $b$ no tienen porque ser finitos, es decir, puede tratarse de intervalos de la forma $(- \infty, b)$.
	\end{enumerate}
\end{observation}
\begin{theorem}
	Dada la ecuación diferencial $x' = f(t, x)$, con $f: [a, b] \times \mathbb{R}^n \longrightarrow \mathbb{R}^n$. Si se cumple una de las siguientes condiciones:
	\begin{enumerate}
		\item $f$ continua en $A = [a, b] \times \mathbb{R}^n$, $f \in L_k (A, x)$.
		\item $f$ continua en $A$, $f \in L_{loc} (A, x)$.
	\end{enumerate}
	entonces, fijado $(t_0, x_0) \in A$ la solución maximal pasando por ese punto está definida en $[a, b]$. 
\end{theorem}
\begin{proof}\ \\
	Definimos la siguiente función, donde $A \subset (c, d) \times \mathbb{R}^n$:
	\[g: (t, x) \in (c, d) \times \mathbb{R}^n \longrightarrow g(t, x) = \begin{cases}
	f(a, x), &t \in (c, a] \\
	f(t, x), &t \in [a, b] \\
	f(b, x), &t \in [b, d] 
	\end{cases}\]
	y con $g$ continua. Esta función verifica las hipótesis del teorema anterior en $(c, d) \times \mathbb{R}^n$ ya que:
	\begin{enumerate}[-]
		\item Si $f \in L_k(A, x) \Rightarrow g \in L_k ((c, d) \times \mathbb{R}^n)$
		\item Si $f$ acotado y $f \in L_{loc} (A, x)$, entonces $g$ acotada y $g \in L_{loc} ((c, d) \times \mathbb{R}^n, x)$.
	\end{enumerate}
	Por tanto, fijado $(t_0, x_0) \in A \subset (c, d) \times \mathbb{R}^n$ existe la solución maximal pasando por $(t_0, x_0)$ y definida en $(c, d)$. Por tanto, $\varphi = \psi|_{[a, b]}$ es la solución maximal de $x' = f(t, x)$ pasando por $(t_0, x_0)$.
	
	DIBUJO
	\end{proof}
BEGIN PRADA 13/03

\begin{proposition}
	Sean $ A $ un abierto de $ \mathbb{R}^{n+1} $ y $ f $ la aplicación continua $ f:(t,x)\in A\subset \mathbb{R}\times \mathbb{R}^n \longrightarrow f(t,x)\in \mathbb{R}^n $, con $  x'=f(t,x) $ su ecuación diferencial asociada. 
	
	Si $ \varphi_1, \varphi_2 $ son soluciones de la ecuación, definidas en $ [a, t_0]$ y $ [t_0,b] $, respectivamente, y además $ \varphi_1(t_0)=\varphi_2(t_0) $, entonces $ \varphi $ definida en $ [a,b] $ como:
	\[ \varphi(t)=\begin{cases}
	\varphi_1(t), \quad t\in [a,t_0]\\
	\varphi_2(t), \quad t\in [t_0,b]
	\end{cases} \]
	es solución de la ecuación diferencial.
\end{proposition}
\begin{proof}\ \\
	Veamos que $ \varphi $ es solución por definición:
	\begin{enumerate}[\quad i)]
		\item $ (t,\varphi(t))\in A \ \forall t \in [a,b], $ ya que $ \begin{cases}
		\forall t\in [a,t_0] \quad (t, \varphi_1(t)) \in A, \text{ ya que }\varphi_1 \text{ es solución.}\\
		\forall t\in [t_0,b] \quad (t, \varphi_2(t)) \in A, \text{ ya que }\varphi_2 \text{ es solución.}\\
		\text{Además, } \varphi_1(t)=\varphi_2(t).
		\end{cases}$
		\item $ \forall t \in [a,t_0) \ \exists \varphi_1'(t)$
		
		$ \forall t \in [t_0,b) \ \exists \varphi_2'(t) $
		
		Veamos si existe la derivada en $ t_0 $, utilizando la continuidad de $ \varphi_1 $ y $ \varphi_2 $:
		\[ \lim\limits_{t \to t_0^-} \varphi'_1(t)=\lim\limits_{t \to t_0^-} f(t, \varphi_1(t))=f(t_0,\varphi_1(t_0)) \]
		\[ \lim\limits_{t \to t_0^+} \varphi'_2(t)=\lim\limits_{t \to t_0^+} f(t, \varphi_2(t))=f(t_0,\varphi_2(t_0)) \]
		Como $ \varphi_1(t_0)=\varphi_2(t_0) \Rightarrow \lim\limits_{t \to t_0^-} \varphi'(t)=\lim\limits_{t \to t_0^+} \varphi'(t)$, y $\varphi $ continua en $ t_0 $, entonces efectivamente existe $ \varphi'(t_0) $.
		\item Inmediata visto lo anterior.
	\end{enumerate} 
\end{proof}
\begin{proposition}
	Sean $ A $ un abierto de $ \mathbb{R}^{n+1} $ y $ f $ la aplicación continua $ f:(t,x)\in A\subset \mathbb{R}\times \mathbb{R}^n \longrightarrow f(t,x)\in \mathbb{R}^n $, con $  x'=f(t,x) $ su ecuación diferencial asociada. 
	
	Entonces, toda solución maximal está definida en un abierto.
\end{proposition}
\begin{proof}\ \\
	Probémoslo por reducción al absurdo. Supongamos que está definida en un intervalo de la forma $ (a,b] $.
	Como $ A $ es abierto, $ (b,\varphi(b))\in A \Rightarrow (b, \varphi(b)) \in \mathring{A} $.
	
	Estamos en la hipótesis del Teorema de Cauchy-Peano. Por lo tanto, pasando por $ (b, \varphi(b)) $ existe solución. Podemos construir una prolongación que sea solución (por el apartado a) y entonces no es maximal.
\end{proof}

\pagebreak
\section{Tema 4.}
\subsection{Métodos elementales de integración de las ecuaciones de primer orden.}
\subsubsection{EDOs de variables separadas.}
Son ecuaciones de la forma $ x'=\frac{\partial x}{\partial t}=h(t)g(x) $. 
\paragraph{Método de resolución}
\begin{enumerate}
	\item $ \frac{1}{g(x)}\cdot\frac{\partial x}{\partial t}=h(t) $
	\item $ \frac{1}{g(x)}\cdot\partial x=h(t)\partial t $
	\item $ \int \frac{1}{g(x)} \partial x= \int h(t) \partial t \Rightarrow G(x)+c_1=H(t)+c_2 \Rightarrow G(x)=H(t)+c$
\end{enumerate}
Si se pide la solución pasando por $ (t_0,x_0) $, hallamos la $ c $ que cumple $ G(x_0)=H(t_0)+c $, es decir, $ G(x)=H(t)+G(x_0)-H(t_0) \Rightarrow G(x)-G(x_0)=H(t)-H(t_0)$.

Otra forma de llegar a este resultado es observar:
\[ \int_{x_0}^{x} \frac{1}{g(x)} \partial x= \int_{t_0}^{t} h(t) \partial t \Rightarrow G(x)-G(x_0)=H(t)-H(t_0)\]
\subsubsection{EDOs homogéneas.}
Son ecuaciones de la forma $ x'=g(\frac{x}{t}), $ o bien $ x'=\frac{f(t,x)}{h(t,x)} $, con $ f,h $ funciones homogéneas del mismo grado.
\begin{definition}
	$ f $ homogénea de grado $ k $ si $ \forall r \in \mathbb{R}, f(rt,rx)=r^kf(t,x)$.
\end{definition}
\paragraph{Método de resolución}
Sean $ f,h $ homogéneas del mismo grado. Todas las homogéneas se pueden dejar como $ x'=g(\frac{x}{t})$:
Reducción de  $ x'=\frac{f(t,x)}{h(t,x)} $: REVISAR
	\[ x'= \frac{f(t,x)}{h(t,x)}=\frac{r^kf(rt,rx)}{r^kh(rt,rx)}=\frac{f(rt,rx)}{h(rt,rx)}=\frac{f(1,\frac{x}{t})}{h(1,\frac{x}{t})}=g(\frac{x}{t})\]
	Una vez tenemos la ecuación en la forma $ x'=g(\frac{x}{t}) $:
	\begin{enumerate}
		\item $ \frac{x}{t}=u\Rightarrow x=u\cdot t $.
		\item $ x'(t)=u(t)+tu'(t)=g(u) \Rightarrow tu'(t)=g(u)-u(t) $
		\item Resolver $ u'(t)=\frac{g(u)-u}{t} $, que es una EDO de variables separadas.
	\end{enumerate}

\subsubsection{EDOs reducibles a homogéneas.}


\subsection{Resolución de ecuaciones diferenciales ordinarias por medio de series de potencias.}

\pagebreak
\section{Tema 5. Sistemas de ecuacións lineais. Propiedades das solucións. Matriz fundamental.}
BEGIN JORGE 18/03

BEGIN PRADA 20/03
\begin{proof}\ \\
	$ F:\varphi\in S\longrightarrow F(\varphi)=\varphi(t_=)\in \mathbb{R}^n $, para un $ t_0\in I $ fijado. 
	Tenemos que demostrar:
	\begin{enumerate}
		\item $ S $ espacio vectorial. 
		\begin{enumerate}[\quad i)]
			\item $ \varphi_1, \varphi_2 \in S \Rightarrow \varphi_1 + \varphi_2 \in S$.
			
			$ (\varphi_1 + \varphi_2)' =\varphi_1'+\varphi_2'=A(t)\varphi_1 +A(t)\varphi_2 = A(t)(\varphi_1 + \varphi_2)\Rightarrow (\varphi_1+\varphi_2)\in S$
			\item $ \lambda\in \mathbb{R}, \varphi\in S $
			
			$ (\lambda \varphi)'=\lambda \varphi'=\lambda A(t)\varphi=A(t)(\lambda \varphi)\Rightarrow \lambda\varphi\in S $.
		\end{enumerate}
		\item $ F $ isomorfismo lineal de $ S $ sobre $ \mathbb{R}^n $.
		\begin{enumerate}[\quad i)]
			\item $ F $ inyectiva. 
			
			Sean $ \varphi_1,\varphi_2 \in S $ tal que $ F(\varphi_1)=F(\varphi_2) $, con $ x_0=\varphi_1(t_0)=\varphi_2(t_0) $.
			
			$ \varphi_1 $ solución del sistema homogéneo $ x'=A(t)x $ pasando por $ (t_0,x_0) $.
			
			$ \varphi_2 $ solución del sistema homogéneo $ x'=A(t)x $ pasando por $ (t_0,x_0) $.
			
			Entonces por unicidad de solución pasando por $ (t_0,x_0) \Rightarrow \varphi_1=\varphi_2\Rightarrow F$ inyectiva.
			\item $ F $ sobreyectiva. 
			
			Sea $ z \in \mathbb{R}^n $ y veamos que $ \exists \varphi \in S : F(\varphi)=z $. Tomo como $ \varphi $ la solución que pasa por $ (t_0, z) $ y por tanto $ F(\varphi)=\varphi(t_0)=z $.	
			\item $ F $ lineal. 
			\[  F(\varphi_1+\varphi_2)=(\varphi_1+\varphi_2)(t_0)=\varphi_1(t_0)+\varphi_2(t_0)=F(\varphi_1)+F(\varphi_2)  \]
			\[ F(\lambda\varphi)=(\lambda\varphi)(t_0)=\lambda \varphi(t_0)= \lambda F(\varphi), \quad \lambda \in \mathbb{R}, \varphi \in S \]	
			
		\end{enumerate}
		Por lo tanto $ F $ es un isomorfismo lineal de $ S $ sobre $ \mathbb{R}^n \Rightarrow \dim_\mathbb{R}S=\dim_{\mathbb{R}^n}\mathbb{R}^n=n $
	\end{enumerate}
\end{proof}
\begin{observation}
	Si $ \{e_1, \dots, e_n\} $ base de $ \mathbb{R}^n $, $ \{ F^1(e_1), \dots F^n(e_n) \} $ base de $ S $. Es decir:
	
	$ F^i(e_i)=\varphi_i $ solución de $ x'=A(t)x $ que pasa por $ (t_0, e_i) $.
\end{observation}
\begin{definition}
	Sea $ \phi(t) \in M_{n\times n}(\mathbb{R}), t\in I$, diremos que $ \phi $ es una matriz solución para el sistema homogéneo $ x'=A(t)x $ si cada columna es solución de él. 
	\[ \begin{pmatrix}
	\vdots & \vdots  \\
	\varphi_1  & \varphi_2 & \dots \\
	\vdots & \vdots  
	\end{pmatrix}  \]
\end{definition}
\begin{exercise}
	$ \phi(t) $ matriz solución para $ x'=A(t)x \Leftrightarrow \phi'(t)=A(t)\phi(t) $
\end{exercise}
\begin{definition}
	Dada $ \phi(t)\in M_{n\times n}(\mathbb{R}) $ matriz solución, decimos que esta es fundamental para $ x'=A(t)x $ si sus columnas forman una base del espacio de soluciones $ S $. Nótese que no es única, ya que las bases tampoco lo son.
\end{definition}
\begin{theorem}[Caracterización de matrices fundamentales]
	Si $ \phi(t) $ es una matriz solución para $ x'=A(t)x $, entonces son equivalentes:
	\begin{enumerate}[\quad i)]
		\item $ \phi(t) $ matriz fundamental.
		\item $ \det(\phi(t))\neq 0, \ \forall t \in I $.
		\item $ \exists t_0 \in I, \det(\phi(t_0))\neq 0 $
	\end{enumerate}
\end{theorem}
\begin{proof}\ \\
	\begin{enumerate}
		\item $ i) \Rightarrow ii) $. \\
		\begin{enumerate}[\quad a)]
			\item Forma 1: \\
			Recordemos que $ \phi (t)= \begin{pmatrix}
			\varphi_1(t) & ... & \varphi_n(t)
			\end{pmatrix} $
			
			Supongamos que existe $ t_1 $ tal que  $ \det(\phi)(t_1)=0\Rightarrow \{ \varphi_1(t_1), \cdots \varphi_n (t_1) \} $ son vectores de $ \mathbb{R}^n  $ linealmente dependientes. Es decir, existen $ \lambda_1, \cdots, \lambda_n \in \mathbb{R}$ no todos nulos tal que $ \lambda_1 \varphi_1 + \cdots +\lambda_n\varphi_n=0 $.
			
			Sea $ \psi(t)=\lambda_1 \varphi_1(t) + \cdots +\lambda_n\varphi_n(t) $. $ \psi $ es solución de $ x'=A(t)x $ (por la estructura de espacio vectorial de $ S $) pasando por $ (t_1,\psi(t_1))=(t_1,0) $.
			
			Pero $ x(t)=0 $ es solución de $ x'=A(t)x$ pasando por $ (t_1,0) $.
			Por la unicidad de la solución $ \psi(t)=0 \ \forall t \in I$. Es decir, $ \lambda_1 \varphi_1(t) + \cdots +\lambda_n\varphi_n(t) =0 \ \forall t \in I $. Como   $ \lambda_1, \cdots, \lambda_n \in \mathbb{R}$ no son  todos nulos, tenemos que $ \phi(t) $ no es matriz fundamental.
			\item Forma 2 (Reducción al absurdo): \\
			Si $ \phi(t) $ es matriz fundamental, entonces $ \phi(t)=(\varphi_1(t), ... , \varphi_n(t)) $ base de $ S $.
			Veamos si xiste un $ t_1 \in I : \det(\phi(t_1))=0$.
			Entonces $ F:\varphi\in S\longrightarrow \varphi(t_1)\in  \mathbb{R}^n $ isomorfismo lineal, es decir, $ \{ \varphi_1(t_1), ... , \varphi_n(t_n) \}  $ base de $ \mathbb{R}^n $ pero no puede ser ya que $ \det(\phi(t_1))=0 \Rightarrow \{ \varphi_1(t_1), ... , \varphi_n(t_n) \} $ son linealmente dependientes.
		\end{enumerate}
	\item $ ii) \Rightarrow iii) $. \\
	Trivial
	\item $ iii) \Rightarrow i) $. \\
	Supongamos que existe un $ t_0 \in I : \det \phi (t_0)\neq 0 $. Por lo tanto, las columnas de $ \phi(t_0) $ son linealmente independientes y forman una base de $ \mathbb{R}^n $.
	
	Sea el isomorfismo $ F:\varphi\in S\longrightarrow \varphi(t_0)\in \mathbb{R}^n $. 
	
	$ \qquad F^{-1}(\varphi_1(t_0))=\varphi_1 $ es solución de $ x'=A(t)x $ pasando por $ (t_0,\varphi_1(t_0)) $ 
	
	$ \qquad \qquad \vdots $ 
	
	$ \qquad F^{-1}(\varphi_n(t_0))=\varphi_n $ es solución de $ x'=A(t)x $ pasando por $ (t_0,\varphi_n(t_0)) $ \\
	Por tanto, $ \{ \varphi_1, \cdots, \varphi_n \} $ base de $ S $ y $ \phi(t) $ es matriz fundamental. 
	\end{enumerate}
\end{proof}
\begin{definition}
	Si $ \phi(t) $ es matriz fundamental para $ x'=A(t)x $ diremos que es principal en $ t_0\in I $ si $ \phi(t_0)=I_n $
\end{definition}
\begin{proposition}[Solución general del sistema homogéneo]
	Sea $ x'= A(t)x, A$ continua en $ I $. Si $ \phi(t) $ es matriz fundamental para $ x'=A(t)x $, entonces las soluciones del sistema son de la forma $ x(t)=\phi(t)c, \ c\in \mathbb{R}^n $.	
\end{proposition}
\begin{proof}\ \\
	$ \phi(t)= (\varphi_1 | \varphi_2 | \dots | \varphi_n) $ matriz fundamental.
	
	$ \{\varphi_1, \varphi_2, \dots,  \varphi_n \} $ base de $ S $.
	
	Si $ \varphi $ es solución, $ \varphi = c_1\varphi_1(t)+ \dots + c_n \varphi_n(t) $, con $ c_1, \dots, c_n\in \mathbb{R} $, lo cual se puede escribir de forma matricial.
	
	Por tanto, $ x(t)=\phi(t)\phi^{-1}(t_0)x_0 $.
\end{proof}
¿Que ocurre si cambiamos de base?
\begin{theorem}
	$ \phi(t) $ matriz fundamental para $ x'=A(t)x $.
	
	$ \psi(t) \in M_{n\times n}(\mathbb{R})$ es matriz fundamental para SH $ \Leftrightarrow \exists \varOmega \in M_{n\times n}(\mathbb{R})$ constante y no singular tal que $ \psi(t) = \phi(t)\varOmega $.
\end{theorem}
\begin{proof}\ \\
	"$ \Leftarrow $": Veamos que $ \psi(t)=\phi(t)\varOmega $ es matriz fundamental:
	\begin{enumerate}[\quad i)]
		\item $ \psi(t)=\phi(t)\varOmega $ es solución de $ x'=A(t)x $ pues $ (\varphi(t))'=(\phi(t)\varOmega)'=\phi'(t)\varOmega=A(t)\phi(t)\varOmega =A(t)\psi $
		\item $ \psi(t) $ es fundamental.
		
		$ \det (\psi(t))=\det(\phi(t))\det(\varOmega)\neq 0 $, por ser $ \phi $ fundamental y $ \varOmega $ no singular. Es decir, $ \psi(t) $ fundamental.
	\end{enumerate}
\end{proof}
\end{document}
