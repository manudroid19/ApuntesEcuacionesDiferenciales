

%-----------------------------------------------------------------------------------------------------  
%	INCLUSIÓN DE PAQUETES BÁSICOS.
%-----------------------------------------------------------------------------------------------------

\documentclass{article}
\usepackage{enumerate}
\usepackage{wrapfig}
\usepackage{graphicx}
\usepackage{tikz}
\usepackage{wrapfig}
\usepackage{cutwin}
\usetikzlibrary{quotes,angles}

\usetikzlibrary{decorations.pathmorphing}


%-----------------------------------------------------------------------------------------------------
%	SELECCIÓN DEL LENGUAJE
%-----------------------------------------------------------------------------------------------------

% Paquetes para adaptar Látex al Español:
\usepackage[spanish,es-noquoting, es-tabla, es-lcroman]{babel} % Cambia
\usepackage[utf8]{inputenc}                                    % Permite los acentos.
\selectlanguage{spanish}                                       % Selecciono como lenguaje el Español.

%flechassssss
\def\flechaSobreyectiva{\mathrel{\mkern16mu  \vcenter{\hbox{$\scriptscriptstyle+$}}%
		\mkern-25mu{\longrightarrow}}}
\def\flechaInyectiva{\mathrel{\mkern0mu  \vcenter{\hbox{$\scriptscriptstyle+$}}%
		\mkern-9mu{\longrightarrow}}}
\def\flechaBiyectiva{\mathrel{\mkern16mu  \vcenter{\hbox{$\scriptscriptstyle+$}}\mkern-25mu{\mathrel{\mkern0mu\vcenter{\hbox{$\scriptscriptstyle+$}}\mkern-9mu{\longrightarrow}}}}}
\def\xFlechaInyectiva #1{\mathrel{\ooalign{\thinspace\thinspace$\mapstochar\mkern5mu$\hfil\cr$\xrightarrow{#1}$\cr}}}
\def\xFlechaSobreyectiva #1{\mathrel{\ooalign{\hfil$\mapstochar\mkern5mu$\thinspace\thinspace\cr$\xrightarrow{#1}$\cr}}}
%llave a la derecha
\newenvironment{rcases}
{\left.\begin{aligned}}
	{\end{aligned}\right\rbrace}

%-----------------------------------------------------------------------------------------------------
%	SELECCIÓN DE LA FUENTE
%-----------------------------------------------------------------------------------------------------

% Fuente utilizada.
\usepackage{courier}                    % Fuente Courier.
\usepackage{microtype}                  % Mejora la letra final de cara al lector.

%-----------------------------------------------------------------------------------------------------
%	ESTILO DE PÁGINA
%-----------------------------------------------------------------------------------------------------

% Paquetes para el diseño de página:
\usepackage{fancyhdr}               % Utilizado para hacer títulos propios.
\usepackage{lastpage}               % Referencia a la última página. Utilizado para el pie de página.
\usepackage{extramarks}             % Marcas extras. Utilizado en pie de página y cabecera.
\usepackage[parfill]{parskip}       % Crea una nueva línea entre párrafos.
\usepackage{geometry}               % Asigna la "geometría" de las páginas.

\providecommand{\norm}[1]{\lVert#1\rVert}
% Se elige el estilo fancy y márgenes de 3 centímetros.
\pagestyle{fancy}
\geometry{left=3cm,right=3cm,top=3cm,bottom=3cm,headheight=1cm,headsep=0.5cm} % Márgenes y cabecera.
% Se limpia la cabecera y el pie de página para poder rehacerlos luego.
\fancyhf{}

% Espacios en el documento:
\linespread{1.1}                        % Espacio entre líneas.
\setlength\parindent{0pt}               % Selecciona la indentación para cada inicio de párrafo.

% Cabecera del documento. Se ajusta la línea de la cabecera.
\renewcommand\headrule{
	\begin{minipage}{1\textwidth}
		\hrule width \hsize
	\end{minipage}
}

% Texto de la cabecera:
\lhead{\docauthor}                          % Parte izquierda.
\chead{}                                    % Centro.
\rhead{IEDO}              % Parte derecha.

% Pie de página del documento. Se ajusta la línea del pie de página.
\renewcommand\footrule{
	\begin{minipage}{1\textwidth}
		\hrule width \hsize
	\end{minipage}\par
}

\lfoot{}                                                 % Parte izquierda.
\cfoot{}                                                 % Centro.
\rfoot{Página\ \thepage\ de\ \protect\pageref{LastPage}} % Parte derecha.

%----------------------------------------------------------------------------------------
%	MATEMÁTICAS
%----------------------------------------------------------------------------------------

% Paquetes para matemáticas:
\usepackage{amsmath, amsthm, amssymb, amsfonts, amscd} % Teoremas, fuentes y símbolos.

\usepackage[all]{xy} %para diagramas conmutativos

% Nuevo estilo para definiciones
\newtheoremstyle{definition-style} % Nombre del estilo
{5pt}                % Espacio por encima
{5pt}                % Espacio por debajo
{}                   % Fuente del cuerpo
{}                   % Identación: vacío= sin identación, \parindent = identación del parráfo
{\bf}                % Fuente para la cabecera
{.}                  % Puntuación tras la cabecera
{.5em}               % Espacio tras la cabecera: { } = espacio usal entre palabras, \newline = nueva línea
{}                   % Especificación de la cabecera (si se deja vaía implica 'normal')

% Nuevo estilo para teoremas
\newtheoremstyle{theorem-style} % Nombre del estilo
{5pt}                % Espacio por encima
{5pt}                % Espacio por debajo
{\itshape}           % Fuente del cuerpo
{}                   % Identación: vacío= sin identación, \parindent = identación del parráfo
{\bf}                % Fuente para la cabecera
{.}                  % Puntuación tras la cabecera
{.5em}               % Espacio tras la cabecera: { } = espacio usal entre palabras, \newline = nueva línea
{}                   % Especificación de la cabecera (si se deja vaía implica 'normal')

% Nuevo estilo para ejemplos y ejercicios
\newtheoremstyle{example-style} % Nombre del estilo
{5pt}                % Espacio por encima
{5pt}                % Espacio por debajo
{}                   % Fuente del cuerpo
{}                   % Identación: vacío= sin identación, \parindent = identación del parráfo
{\scshape}                % Fuente para la cabecera
{:}                  % Puntuación tras la cabecera
{.5em}               % Espacio tras la cabecera: { } = espacio usal entre palabras, \newline = nueva línea
{}                   % Especificación de la cabecera (si se deja vaía implica 'normal')

% Teoremas:
\theoremstyle{theorem-style}  % Otras posibilidades: plain (por defecto), definition, remark
\newtheorem{theorem}{Teorema}[section]  % [section] indica que el contador se reinicia cada sección
\newtheorem{corollary}[theorem]{Corolario} % [theorem] indica que comparte el contador con theorem
\newtheorem{lemma}[theorem]{Lema}
\newtheorem{proposition}[theorem]{Proposición}

% Definiciones, notas, conjeturas
\theoremstyle{definition}
\newtheorem{definition}{Definición}[section]
\newtheorem{conjecture}{Conjetura}[section]
\newtheorem*{note}{Nota} % * indica que no tiene contador
\newtheorem*{observation}{Observación} % * indica que no tiene contador
\newtheorem*{properties}{Propiedades}
\newtheorem*{comment}{Comentario clase}

% Ejemplos, ejercicios
\theoremstyle{example-style}
\newtheorem{example}{Ejemplo}[section]
\newtheorem{exercise}{Ejercicio}[section]

%-----------------------------------------------------------------------------------------------------
%	PORTADA
%-----------------------------------------------------------------------------------------------------

% Elija uno de los siguientes formatos.
% No olvide incluir los archivos .sty asociados en el directorio del documento.
%\usepackage{title1}
\usepackage{title2}
%\usepackage{title3}

%-----------------------------------------------------------------------------------------------------
%	TÍTULO, AUTOR Y OTROS DATOS DEL DOCUMENTO
%-----------------------------------------------------------------------------------------------------

% Título del documento.
\newcommand{\doctitle}{}%git clone https://github.com/manudroid19/ApuntesEstructuras.git
% Subtítulo.
\newcommand{\docsubtitle}{}
% Fecha.
\newcommand{\docdate}{5 \ de \ Noviembre \ de \ 2018}
% Asignatura.
\newcommand{\subject}{Introducción a las Ecuaciones Diferenciales Ordinarias}
% Autor.
\newcommand{\docauthor}{Manuel de Prada, Jorge Vázquez}
\newcommand{\docaddress}{Universidade de Santiago de Compostela}
\newcommand{\docemail}{lalala@gmail.com}

%-----------------------------------------------------------------------------------------------------
%	RESUMEN
%-------------------------------					----------------------------------------------------------------------

% Resumen del documento. Va en la portada.
% Puedes también dejarlo vacío, en cuyo caso no aparece en la portada.
%\newcommand{\docabstract}{}
\newcommand{\docabstract}{Apuntes de la materia impartida en la USC}

\begin{document}

%\maketitle

%-----------------------------------------------------------------------------------------------------
%	ÍNDICE
%-----------------------------------------------------------------------------------------------------

% Profundidad del Índice:
%\setcounter{tocdepth}{1}

\newpage
\tableofcontents
\newpage

%----------------------------------------------------------------------------------------
%	Sección 1: Deficiones y teoremas
%----------------------------------------------------------------------------------------

\section{Tema 1.}

\subsection{Motivaciones, generalidades y ejemplos de ecuaciones diferenciales ordinarias.}
\begin{example}
	Un tren viaja a 90 km/h. A las 10:00 está en el kilómetro 22. ¿Dónde está a las 12:30?
\end{example}
\begin{proof}[Solución]
	Tomando $x(t)$ como la posición en el instante $t$, sabemos que $ x'(t)=90 $, por lo que $ x(t)=90t+k $. Además, $ x(10)=22 $, por lo que con la ecuación anterior, $ k=-878 $. 
	
	Es decir, $ x(t)=90t-878 $ y en particular $ x(12,5)= 247 $.
\end{proof}

\begin{example}
	Un tren está parado en una ciudad A y avanza con una velocidad $ v(t)=2170t+20t^3 $ en m/min hasta que llega a la velocidad de crucero, que es de 4500 m/min. Se pide hallar:
	\begin{enumerate}[a)]
		\item Distancia recorrida en 1 minuto.
		\item Tiempo que tarda en alcanzar la velocidad de crucero.
		\item Metros recorridos cuando se alcanza la velocidad de crucero.
		\item Kilómetros recorridos en 30 minutos.
		\item Hora a la que pasará por la ciudad B, situada a 243 km de distancia, saliendo a las 7:00.
	\end{enumerate}
\end{example}
\begin{proof}[Solución]
	b) Sea $x(t)$ la posición en el instante $t$. Tomando $v(t)=x'(t)=2170t+20t^3=4500$, despejando resulta $ t=2 $. 
	
	a) Buscamos $ x(1) $, para lo que primeros obtendremos la expresión general $ x(t) $ que cumpla la condición $ x'(t)=2170t+20t^3 $.
	
	La primitiva de esta expresión es $ x(t)=\frac{2170t^2}{2}+\frac{20t^4}{4}+k $. Como $ x(0)=0 $, $ k=0 $ y $ x(1)=1090 $.
	
	c) Trivial a partir de los apartados a) y b). $ x(2)=4420 $m.
	
	d) En los dos primeros minutos, hasta alcanzar la velocidad de crucero ya hemos visto que se recorren 4420m. Para los 28 restantes, la velocidad es constante por lo que planteamos la ecuación $ x'(t)=4500 $. 
	
	Por tanto, $ x(t)=4500t+k $. Sabiendo que $ x(2)=4420 $, resulta $ k=-4580 $ y $ x(t)=4500t-4580 $ para $ t>2 $. En particular, $ x(30)=130420 $m $ =130,420 $km.
	
	e) Sabemos por el apartado anterior que se alcanza el punto D (130420m) a los 30 minutos. Los 112,58 metros restantes se recorren en $ \frac{112,58}{4,5\text{m/min}}\simeq 25 $ minutos, por lo que el trayecto en total requiere de 55 minutos y la hora de llegada es las 7:55. 
	
	Otra manera rápida es despejar $ t $ en la ecuación $ x(t)= 243000=4500t-4580 $.
\end{proof}
\begin{example}
	$ x(t) $ es la cantidad de medicamento presente en el organismo en un instante $ t $. La cantidad disminuye proporcionalmente a la cantidad de producto presente en el organismo. En $ t_0 $, tomamos 500mg. En 1 hora, la cantidad de medicamento en el organismo es de 380mg.
	\begin{enumerate}[a)]
		\item ¿Qué cantidad quedará en el cuerpo en 6h?
		\item ¿Cada cuanto tiempo hay que tomar el medicamento para que haya entre 100mg y 700mg en el organismo? 
	\end{enumerate}
\end{example}
\begin{proof}[Solución]
a) Como la cantidad disminuye proporcionalmente a la cantidad de medicamento en el cuerpo, tenemos que si $ x(t) $ es la cantidad de medicamento en un instante $ t $, $ x'(t)=-kx(t) $. Además sabemos que $ x(0)=500 $ y $ x(1)=380 $. Para obtener la expresión correspondiente, razonamos:

Si $ x'(t)=x(t) $, la única opción es que se trate de la exponencial. Nuestra expresión es parecida, tratemos de cumplir la igualdad $ x'(t)=-kx(t) $. Observamos que se cumple para $ x(t)=e^{-kt} $. Ahora nos fijamos en el primer punto,  $ x(0)=500 $. Con  $ x(t)=500e^{-kt} $ cumplimos esa condición. Por último, nos queda hallar una $ k $ que cumpla $ x(1)=500e^{-k}=380 $. 

Nos queda $ k=-\log(\frac{38}{50}) $ y por lo tanto $ x(t)=500e^{\log(\frac{38}{50})t} =500(\frac{19}{25})^t$. En consecuencia, $ x(6)=96,35 $mg.

b) Queremos obtener el $ t $ que cumpla $ 100<x(t)<200 $, para que la dosis no baje de los 100mg prescritos y al tomar de nuevo la dosis de 500mg no supere los 700mg.

$ x(t)=500(\frac{19}{25})^t$ es una exponencial decreciente en todo su dominio, $ (0, +\infty) $. Despejando $ x(t)=500(\frac{19}{25})^t=100$ obtenemos $ t=\log_{\frac{19}{25}}(\frac{1}{5})=5,86 $ y para $ x(t)=200 $, $ t=3,33 $. Por lo tanto, para mantener el medicamento entre los 100mg y 700mg debe renovarse la dosis pasadas entre 3,33 y 5,86 horas.
\end{proof}

BEGIN JORGE 4/2

\subsection{Concepto de solución.}
\begin{definition}
	Sea $f: A \in \mathbb{R} \times \mathbb{R}^n \longrightarrow \mathbb{R}^n$ con $f(t, x)$ una aplicación. Llamamos \underline{Ecuaciones Lineales Ordinarias} (EDO) de primer orden en forma normal relativa a la función $f$ a $x' = f(t, x)$. $\\$
	Se dice que es de primer orden porque solo aparece la primera derivada y forma normal porque $x'$ está despejada (si no sería $h(t, x, x')$).
\end{definition}
\begin{definition}
	Dada $\varphi: t \in I \subset \mathbb{R} \longrightarrow \varphi (t) \in \mathbb{R}^n$ decimos que $\varphi$ es solución de $x'=f(t, x)$ si verifica:
	\begin{enumerate}
		\item $(t, \varphi (t)) \in A$, $\forall t \in I$.
		\item $\exists \varphi' (t)$, $\forall t \in I$.
		\item $\varphi'(t)=f(t, \varphi(t))$,  $\forall t \in I$.
	\end{enumerate}
	Tenemos que si $I$ contiene a uno de sus extremos entenderemos en el punto $2$ que exista su derivada lateral.
\end{definition}
\begin{observation}
	Teniendo que $\varphi: t \in \mathbb{R} \longrightarrow \mathbb{R}^n$ con $\varphi (t) = (\varphi_1 (t), ..., \varphi_n (t))$. Por tanto, que $\varphi' (t) = f(t, \varphi (t))$ significa lo siguiente:
	\[\varphi' (t) = (\varphi_1' (t), ..., \varphi_n' (t))\] 
	lo que se traduce en:
	\[f(t, (x_1, ..., x_n)) = (f_1(t, (x_1, ..., x_n)), ..., f_n(t, (x_1, ..., x_n))) \]
	\[
	M=
	\left({\begin{array}{cc}
		\varphi_1' (t) \\
		\vdots \\
		\varphi_n' (t)
		\end{array} } \right)
	=
	\left({\begin{array}{cc}
		f_1(t, \varphi_1 (t), ..., \varphi_n (t)) \\
		\vdots \\
		f_n(t, \varphi_1 (t), ..., \varphi_n (t))
		\end{array} } \right).
	\]
	Por lo que, tenemos que:
	\[\left\{ {\begin{array}{cc}
		\varphi_1' (t) = f_1(t, \varphi_1 (t), ..., \varphi_n (t)) \\
		\cdots \\
		\varphi_n' (t) = f_n(t, \varphi_1 (t), ..., \varphi_n (t))
		\end{array} } \right. \]
\end{observation}
\begin{proposition}
	Si $f$ es continua en $A$, entonces $\varphi$ es de clase $C^1$ en $I$.
\end{proposition}
\begin{proof}
	Por definición de solución $\varphi' (t) = f(t, \varphi (t))$, $t \in I$. Componiendo de la siguiente forma
	\[t \stackrel{g}{\longrightarrow} (t, \varphi (t)) \stackrel{f}{\longrightarrow} f(t, \varphi (t))\]
	tenemos que, tal y como está definida, $g$ es una función continua por estar compuesta de $id$, que es la identidad, y de $\varphi$, que es continua por ser derivable en todos sus puntos. Además, como $f$ es continua por definición y tenemos que $\varphi' (t) = f(t, \varphi (t))$, $\varphi'$ es continua por composición de continuas.
\end{proof}
\begin{theorem}
	Si $f \in C^1$, $r \in \mathbb{N}$ y $r \leq 1$, entonces $\varphi \in C^{r+1}(I)$.
\end{theorem}
\begin{proof}
	Realizaremos la prueba empleando inducción. En primer lugar lo probaremos para $r=1$, es decir, veamos que:
	\[f\in C^1(A) \Rightarrow \varphi \in C^2(I).\]
	Como $\varphi' (t) = f(t, \varphi (t))$, como hemos visto antes tenemos que:
		\[t \stackrel{g}{\longrightarrow} (t, \varphi (t)) \stackrel{f}{\longrightarrow} f(t, \varphi (t)),\]
	pero, tenemos que $f \in C^1(A)$ por lo que $f$ es continua y por la proposición anterior tenemos que $\varphi \in C^1$. Por lo tanto, la composición anterior cumple que es de clase $C^1$, por lo que concluímos que $\varphi \in C^2$. \\
	Supongamois ahora cierto para un $r$ arbitrario y probémoslo para $r+1$. Sea $f \in C^{r+1} (A)$ (por lo que $f \in C^r \stackrel{HIP}{\Rightarrow}$). Veamos que $\varphi \in C^{r+1}$.
	\[t \longrightarrow (t, \varphi (t)) \longrightarrow f(t, \varphi (t)) = \varphi' (t).\]
	En este caso, por el mismo razonamiento que en el caso $r=1$ tenemos que la composición es de clase $C^{r+1}$ por composición de funciones de clase $C^{r+1}$. Por lo tanto existe $\varphi^{r+2}$ y es continua, por lo que $\varphi \in C^{r+2}$. 
\end{proof}

\begin{example}
	Sea $x' = x^2$, es decir, $f(t, x) \in \mathbb{R} \times \mathbb{R} \longrightarrow f(t,x) = x^2 \in \mathbb{R}$. Comprobar que $x(t) = \frac{-1}{t+c}$, $c \in \mathbb{R}$, es solución. \\
	En primer lugar tenemos que ver que $x'(t) = (x(t))'$.
	\[x'(t) = \frac{1}{(t+c)^2} \Rightarrow x'(t) = (\frac{-1}{t+c})^2\]
	Además, de esta solución sabemos que:
	\[\left. {\begin{array}{cc}
		\lim\limits_{t \to (-c)^{-}} x(t) = \lim\limits_{t \to (-c)^{-}} \frac{-1}{t+c} = +\infty\\
		\lim\limits_{t \to (-c)^{+}} x(t) = \lim\limits_{t \to (-c)^{+}} \frac{-1}{t+c} = -\infty
		\end{array}} \right\} \Rightarrow t = -c \hspace*{0.2cm} AV\]
	por lo que la función tiene una asíntota vertical en $t = -c$. También sabemos que:
	\[\left. {\begin{array}{cc}
		\lim\limits_{t \to +\infty} x(t) = \lim\limits_{t \to +\infty} \frac{-1}{t+c} = 0^-\\
		\lim\limits_{t \to -\infty} x(t) = \lim\limits_{t \to -+\infty} \frac{-1}{t+c} = 0^+
		\end{array}} \right\} \Rightarrow x = 0 \hspace*{0.2cm} AH\]
	por lo que tenemos una asíntota horizontal en $x = 0$. 
	
	%GRAFICA ---------------------------------------------------
	
	En este caso $A = \mathbb{R} \times \mathbb{R} = \mathbb{R}^2$. Además,
	\[t \in (- \infty, - c) \longrightarrow h(t) = \frac{-1}{t+c}\]
	\begin{enumerate}
		\item $(t, h(t)) \in A$
		\item $\exists \varphi' (t) \forall t \in I$
		\item $\varphi (t) = f(t, \varphi (t))$
	\end{enumerate}
	Por lo tanto, $h$ es una solución de $x'=x^2$. Además,
	\[t \in (-c, +\infty ) \longrightarrow \varphi(t) = \frac{-1}{t+c}\]
	también es solución, mientras que:
	\[t \in (- \infty, - c) \cup (-c, +\infty ) \longrightarrow f(t) = \frac{-1}{t+c}\]
	no es solución ya que no está definida en un intervalo. \\
	Además, si queremos calcular la solución de $x'=x^2$ que en $t =1$ valga $3$ tenemos que:
	\[3 = x(1) = \frac{-1}{1+c} \Rightarrow 3 + 3c = -1 \Rightarrow c = -\frac{4}{3}\]
	Por tanto, $x(t) = \frac{-1}{t-\frac{4}{3}}$ verifica que $x'(t) = (x(t))^2$. Además, por lo visto anteriormente, $h$ y $\varphi$ son soluciones si hacemos que $c = - \frac{4}{3}$.
	
	%GRAFICA ------------------------------------------------------
	
	
\end{example}

\subsection{Problema de Cauchy.}

\begin{definition}
	Dada $x' = f(t, x)$ con $f: A \subset \mathbb{R} \times \mathbb{R}^n longrightarrow \mathbb{R}^n$ y un punto $(t_0, x_0) \in A$, el problema de Cauchy consiste en buscar una función:
	\[\varphi : I \subset \mathbb{R} \longrightarrow \mathbb{R}^n\]
	con $I$ un intervalo, que sea solución de $x' = f(t, x)$ tal que $\varphi (t_0) = x_0$.
\end{definition}

BEGIN MANU 6/2 exp

¿Una circunferencia puede ser solución de una EDO?
\[ t^2+x^2=c \]
\begin{wrapfigure}{l}{3.5cm}   
	\begin{tikzpicture}[cap=round,>=latex,every node/.style={scale=0.5}]
	\draw[thick] (0cm,0cm) circle(1cm);
	\node[circle,draw=white, fill=white, inner sep=0pt,minimum size=5pt] (b) at (1cm,0) {};
	\node[circle,draw=white, fill=white, inner sep=0pt,minimum size=5pt] (b) at (-1cm,0) {};
	\draw[->] (-1.5cm,0cm) -- (1.5cm,0cm) node[right,fill=white]{$t$};
	\draw[->] (0cm,-1.5cm) -- (0cm,1.5cm) node[above,fill=white]{};
	\node[above left] at (-1cm,0) {$(-\sqrt{c},0)$};
	\node[above right] at (1cm,0) {$(\sqrt{c},0)$};
	\node[above right] at (0.7cm,0.7cm) {$x_1(t)$};
	\node[below right] at (0.7cm,-0.7cm) {$x_2(t)$};
	\end{tikzpicture}
\end{wrapfigure}
No, ya que no es una función. Sin embargo, si tomamos $ x_1=\sqrt{c-t^2} $ y $ x_1=\sqrt{c-t^2} $, son solución de la ecuación $ xx'+t=0 $.

Veámoslo para $ x_1 $.
\[ (c-t^2)^{\frac{1}{2}}\frac{1}{2} (c-t^2)^{-\frac{1}{2}}(-2t)+t=-t+t=0 \]
Su intervalo de definición será $ c-t^2\geq 0 \Rightarrow t \in [-\sqrt{c}, \sqrt{c}], c>0 $ y es derivable en $ (-\sqrt{c}, \sqrt{c}) $.

La solución escogida será una o otra en función de la condición inicial.
\[ t\in (-\sqrt{c}, \sqrt{c}) \longrightarrow \begin{cases}
x_1(t)=\sqrt{c-t^2}\\
x_2(t)=-\sqrt{c-t^2}
\end{cases}  \]
\pagebreak
\section{Tema 2. Existencia y unicidad de solución.}

\begin{definition}
	Sea $ \varphi:t\in I \subset \mathbb{R}\longrightarrow \mathbb{R}^n $, con $ I $ intervalo, solución de la ecuación diferencial definida por $ f:A \subset \mathbb{R}\times \mathbb{R}^n$, que lleva $(t,x)\longrightarrow f(t,x)$ y tal que $ x'=f(t,x) $.
\end{definition}

Llamamos trayectoria de $ \varphi $ al conjunto $ \tau (\varphi)=\{ (t, \varphi(t)), t\in I\} $

Si $ \varphi(t_0)=x_0 $, llamaremos semitrayectoria positiva de $ \varphi $ al conjunto $ \tau^+ (\varphi)=\{ (t, \varphi(t)) : t\in I, t>t_0\} $

Análogamente, la semitrayectoria negativa  de $ \varphi $ será el conjunto $ \tau^- (\varphi)=\{ (t, \varphi(t)) : t\in I, t\leq t_0\} $.

\begin{definition}
	Sea la ecuación diferencial definida por $ f:A \subset \mathbb{R}\times \mathbb{R}^n$, que lleva $(t,x)\longrightarrow f(t,x)$ y tal que $ x'=f(t,x) $.
	
	Sea $ \varepsilon\in \mathbb{R}, \varepsilon\geq 0 $, y sea $ \pi : I\subset \mathbb{R} \longrightarrow \mathbb{R}^n $ continua. Diremos que $ \pi $ es una solución $ \varepsilon $-aproximada si verifica:
	
	\begin{enumerate}
		\item $ (t,\pi (\varepsilon))\in A, \forall t \in I $.
		\item $ \pi $ continua y $C^1$ a trozos (es decir, es $C^1$ salvo a lo sumo una cantidad finita de puntos)
		\item $ || \pi'(t)-f(t, \pi(t))|| \leq \varepsilon$. (Las normas son equivalentes en dimensión finita).
	\end{enumerate} 
\end{definition}
\begin{proposition}
	Sea la ecuación diferencial definida por $ f:A \subset \mathbb{R}\times \mathbb{R}^n$, que lleva $(t,x)\longrightarrow f(t,x)$ y tal que $ x'=f(t,x) $.
	
	Sea $ f $ continua y sea $ \pi : t \in I \subset \mathbb{R} \longrightarrow\mathbb{R}^n $, entonces:
	
	$ \pi  $ solución $ \Leftrightarrow $ $ \pi $ solución 0-aproximada.
\end{proposition}
\begin{proof}
	"$ \Rightarrow $". Ejercicio propuesto.
	
	
	"$ \Leftarrow $". Suponemos que $ \pi $ solución 0-aproximada y veamos que $ \pi $ es solución.

	1) se verifica trivialmente.
	
	Sea $ \{t_i\}_{i=1}^k $ los puntos en los que $ \pi $ puede, en principio, no ser derivable, y veamos que existe la derivada en ellos.
	
	Por una parte, por ser  $ \pi $ solución 0-aproximada, $ ||\pi'(t)-f(t,\pi(t))||\leq 0\Rightarrow ||\pi'(t)-f(t,\pi(t))||=0 \Rightarrow \pi'(t)=f(t, \pi(t)), \forall t \in I, t\neq t_i$.
	
	Fijado $ t_i\in \{t_1, \dots, t_k\} $, como $ f $ y $ \pi $ son continuas, 
	\[ \lim\limits_{t\rightarrow t_i^-}\pi'(t)=\lim\limits_{t\rightarrow t_i^-} f(t, \pi(t))= f(t_i, \pi(t_i))= \lim\limits_{t\rightarrow t_i^+} f(t, \pi(t))=\lim\limits_{t\rightarrow t_i^+}\pi'(t)\]
	Y por lo tanto:
	\[ \begin{rcases}
	\lim\limits_{t\rightarrow t_i^-}\pi'(t)=\lim\limits_{t\rightarrow t_i^+}\pi'(t) \\
	\pi \text{ continua en } t_i
	\end{rcases}\Rightarrow \exists \pi'(t_i), \forall i \in \{1,\dots,k \} \]
	
\end{proof}

\begin{theorem}
	Sea $ \pi: t  \in I \subset \mathbb{R} \longrightarrow\mathbb{R}^n $, con $ I  $ intervalo, una solución $ \varepsilon $-aproximada de la ecuación diferencial definida por $ f:A \subset \mathbb{R}\times \mathbb{R}^n$, que lleva $(t,x)\longrightarrow f(t,x)$ y tal que $ x'=f(t,x) $.
	
	$ f $ continua, entonces:
	\[ ||\pi(t)-\pi(t_0)-\int_{t_0}^{t}f(s,\pi(s))ds||\leq \varepsilon |t-t_0|, \forall t, t_0 \in I\]
	 
\end{theorem}
\begin{proof} \textbf{Caso I.}
	
	Supongamos $ \pi \in C^1 $ salvo en los puntos $ t_1<\dots<t_k $.
	
	\begin{tikzpicture}[cap=round,>=latex,every node/.style={scale=1}]
	\foreach \x/\y in {0/t_1, 1/t_2, 4/t_k}
	\draw[thick] (\x,0.25) -- (\x,-0.25) node[below]{$\y$};
	\draw[thick] (-0.5,0)--(4.5,0);
	\draw (2.5,-0.25) -- (2.5,-0.25) node[below]{$\dots$};
	\end{tikzpicture}
	
	Supongamos $ t,t_0 \in [t_i, t_{i+1}], i \in \{1, \dots, k-1\}$, es decir:
	
	\begin{tikzpicture}[cap=round,>=latex,every node/.style={scale=1}]
	\foreach \x/\y in {0/t_i, 3/t_{i+1}, 6/t_{i+2}}
	\draw[thick] (\x/1.5,0.25) -- (\x/1.5,-0.25) node[below]{$\y$};
	\draw[very thick] (1/1.5,0.10) -- (1/1.5,-0.10) node[below]{$t_0$};	
	\draw[very thick] (2/1.5,0.10) -- (2/1.5,-0.10) node[below]{$t$};
	\draw[thick] (-0.5,0)--(4.5,0);
	\end{tikzpicture}
	
	Sea $ J=[t_0,t], J \subset [t_i, t_{i+1}] $.
	
	Sea la función 
	\begin{align*}
	g:J&\longrightarrow \mathbb{R}^n\\
	u&\longrightarrow g(u)=\pi(u)-\pi(t_0)- \int_{t_0}^{u}f(s, \pi(s)) ds
	\end{align*}	
	$\begin{rcases}
	 \pi  \text{ continua por ser solución }  \varepsilon \text{-aproximada.}\\
	 s\longrightarrow (s, \pi(s))\longrightarrow f(s, \pi(s)) \text{ es composición de continuas.}\\
	 F:J\longrightarrow \mathbb{R} \text{ que lleva } u\longrightarrow  \int_{t_0}^{u}f(s, \pi(s)) ds \text{ continua.}
	\end{rcases}$ $ \Rightarrow g $ continua en $ J $.
	
	$ \begin{rcases}
	 \pi  \text{ derivable en } \mathring{J} \\
	 F \text{ derivable en } J 
	\end{rcases} $ $ \Rightarrow g $ derivable en $ \mathring{J} $.
	
	$ \begin{rcases}
	 g \text{ derivable en }  J \\
	gF \text{ derivable en } \mathring{J} 
	\end{rcases} $ $ \Rightarrow $ (por th. incrementos finitos) $ ||g(t)-g(t_0)||\leq \sup ||g'(\xi)||  |t-t_0| \Rightarrow $ (def. de $ g $)
	
	\begin{align*}
	\Rightarrow ||\pi(t)-\pi(t_0)-\int_{t_0}^{t} f(s, \pi(s)) ds -(\pi(t_0-\pi(t_0))-\int_{t_0}^{t_0} f(s,\pi(s)) ds)|| =\\ =||\pi(t)-\pi(t_0)-\int_{t_0}^{t} f(s, \pi(s)) ds|| \leq \sup ||g'(\xi)|| |t-t_0|
	\end{align*}  
	Ahora bien, por el Teorema Fundamental del Cálculo y la definición de solución $\varepsilon$-aproximada: 
	\begin{align*}
	g'(u)=\pi'(u)-f(u,\pi(u))\Rightarrow ||g'(u)||=||\pi'(u)-f(u,\pi(u))||\leq \varepsilon 
	\end{align*}
	Y finalmente:
	\[ \begin{rcases}
	||\pi(t)-\pi(t_0)-\int_{t_0}^{t} f(s, \pi(s)) ds|| \leq \sup ||g'(\xi)|| |t-t_0|\\
	||g'(u)||=||\pi'(u)-f(u,\pi(u))||\leq \varepsilon 
	\end{rcases} \Rightarrow	||\pi(t)-\pi(t_0)-\int_{t_0}^{t} f(s, \pi(s)) ds|| \leq \varepsilon |t-t_0|\]
	
	BEGIN JORGE 11/2
	
	\textbf{Caso II.}
	En este caso el caso a analizar va a ser el de dos puntos en dos intervalos diferentes pero consecutivos. Es decir:
	
	\begin{tikzpicture}[cap=round,>=latex,every node/.style={scale=1}]
	\foreach \x/\y in {0/t_i, 3/t_{i+1}, 6/t_{i+2}}
	\draw[thick] (\x/1.5,0.25) -- (\x/1.5,-0.25) node[below]{$\y$};
	\draw[very thick] (1/1.5,0.10) -- (1/1.5,-0.10) node[below]{$t_0$};	
	\draw[very thick] (4/1.5,0.10) -- (4/1.5,-0.10) node[below]{$t$};
	\draw[thick] (-0.5,0)--(4.5,0);
	\end{tikzpicture}
	
	Así, en este caso, tenemos que $t_o \in [t_i, t_{i+1}]$ y que $t \in [t_{i+1}, t_{i+2}]$. Por lo tanto, definiremos $J_j = [t_o, t_{i+1}]$ y $J_r = [t_{i+1}, t]$. Repitiendo el proceso en cada intervalo se tiene que:
	\begin{enumerate}
		\item $\norm{\pi (t) - \pi (t_{i+1}) - \int_{t_{i+1}}^{t} f(s, \pi(s))ds} \leq \varepsilon |t - t_{i+1} |$ en $J_r$.
		\item $\norm{\pi (t_{i+1}) - \pi (t_0) - \int_{t_0}^{t_{i+1}} f(s, \pi(s))ds} \leq \varepsilon |t_{i+1} - t_0|$ en $J_j$.
	\end{enumerate}
	Considerando ahora la suma siguiente:
	\[\pi (t) - \pi (t_{i+1}) - \int_{t_{i+1}}^{t} f(s, \pi(s))ds + \pi (t_{i+1}) - \pi (t_0) - \int_{t_0}^{t_{i+1}} f(s, \pi(s))ds = \pi (t) - \pi (t_0) - \int_{t}^{t_0} f(s, \pi(s))ds \]
	Por otro lado sabemos que:
	\[\norm{\pi (t) - \pi (t_{i+1}) - \int_{t_{i+1}}^{t} f(s, \pi(s))ds + \pi (t_{i+1}) - \pi (t_0) - \int_{t_0}^{t_{i+1}} f(s, \pi(s))ds} \stackrel{Desigualdad Triangular}{\leq} \norm{\pi (t) - \pi (t_{i+1}) - \int_{t_{i+1}}^{t} f(s, \pi(s))ds} + \norm{\pi (t_{i+1}) - \pi (t_0) - \int_{t_0}^{t_{i+1}} f(s, \pi(s))ds} \stackrel{1. y 2.}{\leq}  \varepsilon |t - t_{i+1} | + \varepsilon |t_{i+1} - t_0| = \varepsilon |t-t_0|\]
	Por lo que, como queríamos probar:
	\[\norm{\pi (t) - \pi (t_0) - \int_{t}^{t_0} f(s, \pi(s))ds} \leq \varepsilon |t-t_0|\]
	
	\textbf{Caso III.} Ejercicio propuesto. Basta con realizar inducción y basarse en el caso III.
\end{proof}
\begin{theorem}
	Dada la ecuación (*) con $f$ continua y $\varPhi :t \in I \subset \mathbb{R} \longrightarrow \mathbb{R}^n$ tal que $(t, \varphi (t)) \in A$ $\forall t \in A$ se tiene:
	\begin{center}
		$\varphi$ solución de (*) $\Leftrightarrow \begin{cases}
		(1) \hspace{0,2cm} \varphi \hspace{0,2cm} continua \\
		(2) \hspace{0,2cm} \varphi (t)= \varphi (t_0) + \int_{t_0}^{t} f(s, \pi (s))ds
		\end{cases}$
	\end{center}
\end{theorem}
\begin{proof}
	$"\Rightarrow"$ \\
	Supongamos $\varphi$ solución de $x' = f(t, x) \Rightarrow
	\begin{cases}
	i) \hspace{0,2cm} (t, \varphi (t)) \in A \forall t \in I
	ii) \hspace{0,2cm} \exists \varphi' (t) \forall t \in I
	iii) \hspace{0,2cm} \varphi' (t) = f(t, \varphi (t)) t \in I
	\end{cases}$. \\
	En primer lugar $ii) \Rightarrow \varphi$ continua. Además, por $iii)$ sabemos que $\varphi' (t) = f(t, \varphi (t))$. Vemos que la composición:
	\[t \stackrel{g}{\longrightarrow } (t, \varphi (t)) \stackrel{f}{\longrightarrow } f(t, \varphi (t))\]
	es continua por ser continuas $g$ y $f$, por lo que $\varphi '$ es integrable. Entonces,
	\[\varphi (t) - \varphi (t_0) = \int_{t_0}^{t} \varphi' (s)ds = \int_{t_0}^{t} f(s, \varphi (s))ds \Rightarrow \varphi (t) = \varphi (t_0) + \int_{t_0}^{t} f(s, \varphi (s))ds\]
	$"\Leftarrow"$ \\
	Ahora supongo que se cumplen $(1)$ y $(2)$.
	
\end{proof}


BEGIN MANU 13/2 

\begin{theorem}[Construccion de solución $ \varepsilon $-aproximada.]
	Consideramos la ecuación diferencial definida por $ f:A \subset \mathbb{R}\times \mathbb{R}^n$, que lleva $(t,x)\longrightarrow f(t,x)$ y tal que $ x'=f(t,x) $, con $ A $ abierto, $ f $ continua y acotada en $ A $ y sea $ \varepsilon>0, \varepsilon\in \mathbb{R} $. 
	
	Entonces fijado $ (t_0, x_0) \in A $ existe solución $ \varepsilon $-aproximada pasando por $ (t_0, x_0)$.
\end{theorem}
\begin{proof} \ 
	
	\begin{figure}[h]
		\centering
	\begin{tikzpicture}
	\draw[->] (-1,0) -- (5,0) node[right] {$t \in \mathbb{R}$}; %eje x, en este caso t
	\draw[->] (0,-0.5) -- (0,3.5) node[above] {$x \in \mathbb{R}^n$};% eje y
	\draw  plot[tension=.7] coordinates {(1,1) (1,2.5) (4,2.5) (4,1) (1,1) }; %conjunto R
	\draw  plot[scale=0.53,shift={(4.3,2.75)},smooth, tension=.7] coordinates {(-3.5,0.5) (-3,2.5) (-1,3.5) (1.5,3) (4,3.5) (5,2.5) (5,0.5) (2.5,-2) (0.5,-1) (-3,-2) (-3.5,0.5)}; % conjunto A
	\draw[-] (2.5,0.10) -- (2.5,-0.10) node[below][scale=0.8]{$t_0$}; %t0 y puntos de x
	\draw[-] (1,0.10) -- (1,-0.10) node[below][scale=0.8]{$t_0-\delta$};
	\draw[-] (4,0.10) -- (4,-0.10) node[below][scale=0.8]{$t_0+\delta$};
	\draw[-] (3.0,0.10) -- (3.0,-0.10) node[below][scale=0.8]{$t_1$};
	\draw[-] (-0.10,1.75) -- (0.10,1.75) node[left=2mm][scale=0.8]{$x_0$}; %x_0 y puntos de y
	\draw[-] (-0.10,1) -- (0.10,1) node[left=2mm][scale=0.8]{$x_0-H\delta$};
	\draw[-] (-0.10,2.5) -- (0.10,2.5) node[left=2mm][scale=0.8]{$x_0+H\delta$};
	\coordinate[label=above:A] (A) at (5.2,2.8); %etiqueta de A
	\coordinate[label=above:R] (R) at (4.2,2.2); %etiqueta de R
	\draw  plot[tension=.7] coordinates {(1.25,1.75) (1.5,2.2) (1.7,2) (2.0,2.4) (2.5,1.75) (3.0, 2.3) (3.4, 1.9) (3.7, 2.3)}; %linea poligonal
	\fill (2.5,1.75)  circle[radius=1pt]; %punto en t0,x_0
	\coordinate[label={[scale=0.6] right:$(t_0,x_0)$}] (p0) at (2.5,1.7); %etiqueta de t0 x0
	\coordinate[label={[scale=0.6] above:$\rho_1$}] (rho1) at (2.7,2.0); %etiqueta de rho 1
	\draw[-,dashed] (2.5,0.9275) -- (2.5, 1.75) node[right=1mm][below=3mm][scale=0.6]{$d$}; %linea punteada de distancia
	\end{tikzpicture}
	\caption{Esquema de la demostración.} \label{M1}
\end{figure}
	
	Sea $ d=d_\infty((t_0, x_0), \text{Fr} (A))>0 $ (ya que $ A $ abierto).
	\[ d=\inf \{d_\infty ( (t_0,x_0), (t,x)):(t,x)\in \text{Fr} (A) \}=\inf \{\max \{|t-t_0|,||x-x_0||\}:(t,x)\in \text{Fr} (A)\}\]
	Como $ f $ acotada, $ \exists H $ para $ f $, tal que $ \forall(t,x) \in A, ||f(t,x)||<H $.
	
	Definamos el siguiente conjunto $ R $, con $ \delta $ tal que $ 0<\delta<\min \{d, \frac{d}{H}\} $:
	\[ R=\{(t,x):|t-t_0|<\delta, ||x-x_0||\leq H\delta\} \]
	Veamos que $ R \subset A $. $ (t_1,x_1)\in R \Rightarrow \begin{cases}
	|t_1-t_0|<\delta<\min\{d,\frac{d}{H}\}\leq d\Rightarrow |t_1-t_0|<d\\
	||x_1-x_0||\leq H\delta<H\min\{d,\frac{d}{H}\}\leq H \frac{d}{H}=d \Rightarrow ||x_1-x_0||<d
	\end{cases}$  
	
	$\begin{rcases}
	\text{De estas dos condiciones se deduce } d_\infty ( (t_0,x_0), (t_1,x_1)) <d\\
	A \text{ abierto}\\
	d=d_\infty((t_0, x_0), \text{Fr} (A))
	\end{rcases} \Rightarrow (t_1,x_1)\in A \Rightarrow R \subset A$.
	
	Si existe $ \varphi $ solución de la ecuación diferencial del enunciado, la tangente a $ \varphi $ en $ (t_0,\varphi(t_0))=(t_0,x_0) $ sería (ya que $ \varphi $ es solución):
	\[ \varphi'(t)|_{t_0}=f(t, \varphi(t))|_{t_0}=f(t_0,\varphi (t_0))=f(t_0,x_0) \]
	Definimos entonces la siguiente función, que aproxima mediante una recta a $ \varphi $ en un entorno de $ (t_0,x_0) $: 
	\[ \rho_1:I\subset \mathbb{R}\longrightarrow \mathbb{R}^n \]
	\[ t \mapsto \rho_1(t)=x_0+f(t_0,x_0)(t-t_0) \]
	Es la ecuación de la recta pasando por $ (t_0,x_0) $ con pendiente $ f(t_0,x_0) $.
	
	Veamos para que valores de $ t $, $ \rho_1(t) $ es solución $ \varepsilon $-aproximada. 
	Si $ t\geq t_0 $, la pendiente de $ \rho_1 $ en $ t $ es $ f(t_0,x_0) $.
	Como $ f $ acotada en $ A $, tenemos $ ||f(t_0,x_0)||\leq H $.
	
	Veamos si $ \rho_1 $ verifica las condiciones de ser $ \varepsilon $-aproximada.
	
	i) $ (t,\rho(t)) \in \mathbb{R}\subset A$.
	
	ii) $ \rho_1 $ es continua y $ C^n $ a trozos ya que $ \rho_1 \in C^\infty $.
	
	iii) ¿Se cumple $ ||\rho_1'(t)-f(t,\rho_1(t))||\leq \varepsilon $?
	
	Por una parte, $ ||\rho_1'(t)-f(t,\rho_1(t))||=||f(t_0,x_0)-f(t,x_0+f(t_0,x_0)(t-t_0))||  $
	
	Por otra, $ \begin{rcases}
	f \text{ continua en } A\\
	R\subset A \text{ compacto}
	\end{rcases} \Rightarrow f \text{ uniformemente continua en } R\text{, es decir, } \forall \varepsilon > 0 \ \exists \hat{\delta} >0 : \forall y,z \in R, ||y-z||<\hat{\delta} \Rightarrow||f(y)-f(z)||<\varepsilon $.
	
	Tengo que demostrar que $ ||\rho_1'(t)-f(t,\rho_1(t))||=||f(t_0,x_0)-f(t,x_0)+f(t_0,x_0)(t-t_0)||=||f(y)-f(z)||<\varepsilon $. 
	
	Por lo anterior ($ f $ uniformemente continua), si para $ \varepsilon>0 $ demuestro que $||y-z||=||(t_0,x_0)-(t, x_0+f(t_0,x_0)(t-t_0))||< \hat{\delta} $, ya queda probado.
	
	Veamos para que valores de $ t $, se tiene $ ||(t_0,x_0)-(t, x_0+f(t_0,x_0)(t-t_0))||< \hat{\delta} \Rightarrow ||(t_0-t), x_0-x_0 -f(t_0,x_0)(t-t_0)||=||t_0-t,f(t_0,x_0)(t-t_0)||< \hat{\delta} $.
	
	Notemos que al trabajar con la norma infinito tengo que comprobar que los máximos de los valores absolutos de las componentes son menores que la cota.
	
	\[ \begin{rcases}
	 |t-t_0|< \hat{\delta}\\
	 \text{ y }\qquad\\
	|| -f(t_0,x_0)(t-t_0)||< \hat{\delta}
	\end{rcases} \Rightarrow ||(t_0-t), f(t_0,x_0)(t-t_0)||< \hat{\delta} \]
	Sea $ \alpha >0 $ tal que $ \alpha < \min \{ \hat{\delta}, \frac{\hat{\delta}}{H}\} $ veamos que $ \forall t \in [t_0,t_0+\alpha] $, $\rho_1(t)$ es solución $ \varepsilon $-aproximada, es decir, se cumple $ ||t_0-t,f(t_0,x_0)(t-t_0)||< \hat{\delta}\Rightarrow ||\rho_1'(t)-f(t,\rho_1(t))||\leq \varepsilon $.
	
	\[ t \in [t_0, t_0+\alpha]\Rightarrow |t-t_0|\leq \alpha  
	\begin{rcases}
	< \hat{\delta}\leq\frac{\hat{\delta}}{H} \text{(si } min\{ \hat{\delta}, \frac{\hat{\delta}}{H}\}= \hat{\delta})\ \  \\
	< \frac{\hat{\delta}}{H}<\hat{\delta} \text{(si }  min\{ \hat{\delta}, \frac{\hat{\delta}}{H}\}=  \frac{\hat{\delta}}{H})
	\end{rcases}\Rightarrow \begin{cases}
	|t-t_0|<\hat{\delta}\\
	|t-t_0|<\frac{\hat{\delta}}{H}
	\end{cases}
	 \]
	 Solo falta comprobar si $ || -f(t_0,x_0)(t-t_0)||< \hat{\delta} $:	 
	  \[ t \in [t_0, t_0+\alpha]\Rightarrow ||-f(t_0,x_0)(t-t_0)|| = || f(t_0,x_0)||\cdot|t-t_0|<H\alpha < H \frac{\hat{\delta}}{H}= \hat{\delta}   \]
	Entonces, tenemos visto que $ \rho_1 $ es solución $ \varepsilon $-aproximada en $ [t_0,t_0+\alpha] $. Sea ahora $ t_1=t_0+\alpha $ y $ x_1=\rho_1(t_1) $ y definimos la siguiente función:
	\[ \rho_2:I\longrightarrow \mathbb{R} \]
	\[ t \longmapsto \rho_2(t)=x_1+f(t_1,x_1)(t-t_1) \]
	se ve de forma análoga que $ \rho_2(t) $ es solución $ \varepsilon $-aproximada $  \forall t \in [t_1,t_1+\alpha]  $ y repetimos el proceso construyendo la solución $ \varepsilon $-aproximada en todo $ R $.
\end{proof}
\section{Tema 3.}
\pagebreak
\section{Tema 4.}
\subsection{Métodos elementales de integración de las ecuaciones de primer orden.}
\begin{definition}
	Sea $ f:A\subset \mathbb{R}\times \mathbb{R}^n \longrightarrow \mathbb{R}^n$
	
	Se llama Ecuación Diferencial Ordinaria de primer orden (solo aparece la primera derivada) en forma normal ($ x' $ aparece separada en uno de los miembros de la ecuación) relativa a la función $ f $ a la expresión $ x'=f(t,x) $.
\end{definition}
\begin{definition}
	Dada $ \varphi : t\in I\subset \mathbb{R} \longrightarrow \varphi(t)\in \mathbb{R}^n  $, con $ \varphi $ definida en $ I $ un intervalo cualquiera.
	
	Decimos que $ \varphi $ es solución de $ x'=f(t,x) $ si verifica:
	\begin{enumerate}[i)]
		\item $(t,\varphi(t)) \in A, \forall t \in I$.
		\item $ \exists \varphi'(t), \forall t\in I $. (Si $ I $ contiene a uno de sus extremos entenderemos la condición como la existencia de la derivada lateral)
		\item $ \varphi'(t)=f(t,\varphi(t)), \forall t \in I $.
	\end{enumerate}
\end{definition}
\begin{observation}
	\[ \varphi : t\in\mathbb{R} \longrightarrow\mathbb{R}^n\]
	\[ t \longmapsto (\varphi_1(t), \dots \varphi_n(t)) \]
	Por tanto, $ \varphi'(t)=f(t,\varphi(t)) $ significa lo siguiente: 
	\[ \varphi'(t)=(\varphi'_1(t), \dots,\varphi'_n(t))  \]
	\[ f(t,(x_1, \dots , x_n))=(f_1(t, x_1, \dots , x_n), \dots , f_n(t,x_1, \dots , x_n) \]
	\[ \begin{pmatrix}
	\varphi'_1(t) \\
	\vdots \\
	\varphi'_n(t)
	\end{pmatrix}=
	 \begin{pmatrix}
	 f_1(t, \varphi_1(t), \dots, \varphi_n(t))\\
	 \vdots \\
	 f_n(t,\varphi_1(t), \dots, \varphi_1(t))
	 \end{pmatrix} 
	 \text{, i.e: }\begin{cases}
	 \varphi'_1(t)=f_1(t, \varphi_1(t), \dots, \varphi_n(t))\\
	 \qquad\qquad\qquad\vdots\\
	 \varphi'_n(t)=f_n(t, \varphi_1(t), \dots, \varphi_n(t))
	 \end{cases}\]
\end{observation}
\subsubsection{Método de resolución de EDOs por variables separadas.}
Se utiliza para ecuaciones del tipo $ x'=\frac{\partial x}{\partial t}=h(t)g(x) $. Pasos para resolverlas:
\begin{enumerate}[1)]
	\item $ \frac{1}{g(x)}\cdot\frac{\partial x}{\partial t}=h(t) $
	\item $ \frac{1}{g(x)}\cdot\partial x=h(t)\partial t $
	\item $ \int \frac{1}{g(x)} \partial x= \int h(t) \partial t \Rightarrow G(x)+c_1=H(t)+c_2 \Rightarrow G(x)=H(t)+c$
\end{enumerate}
Si se pide la solución pasando por $ (t_0,x_0) $, hallamos la $ c $ que cumple $ G(x_0)=H(t_0)+c $, es decir, $ G(x)=H(t)+G(x_0)-H(t_0) \Rightarrow G(x)-G(x_0)=H(t)-H(t_0)$.

Otra forma de llegar a este resultado es observar:
\[ \int_{x_0}^{x} \frac{1}{g(x)} \partial x= \int_{t_0}^{t} h(t) \partial t \Rightarrow G(x)-G(x_0)=H(t)-H(t_0)\]

\subsection{Resolución de ecuaciones diferenciales ordinarias por medio de series de potencias.}

\end{document}
